\documentclass{article}
\usepackage[utf8]{inputenc}
\usepackage[T1]{fontenc}
\usepackage[francais]{babel}
\usepackage{hyperref}
\usepackage{parselines}

\hypersetup{
unicode=false, % non-Latin characters bookmarks
pdftoolbar=false, % show Acrobat's toolbar?
pdfmenubar=false, % show Acrobat's menu?
pdffitwindow=true, % page fit to window when opened
pdftitle={Rapport Genie logiciel}, % title
pdfauthor={REYNAUD Nicolas}, % author
pdfsubject={thèse}, % subject of the document
pdfnewwindow=true, % links in new window
colorlinks=true, % false: boxed links; true: colored links
linkcolor=black, % color of internal links
citecolor=green, % color of links to bibliography
filecolor=magenta, % color of file links
urlcolor=blue % color of external links
}

\usepackage{manyfoot}
\usepackage[stable]{footmisc}
\usepackage{geometry}
\usepackage{graphicx}
\usepackage{amssymb}

\geometry{hmargin=3.0cm,vmargin=1.5cm}

\title{Rapport Reunion \#2}
\author{Auteur : Nicolas \bsc{Reynaud}}
\date{14 Octobre 2014}

\parindent=0pt

\begin{document}

\maketitle
\newpage

Présents à cette réunion : Bascol Kevin, Laoussing Kévin, Reynaud Nicolas.

\subsection*{Ordre du jour}
\indent -Décider de la planification\\
-Faire le diagramme de gantt\\

\subsection*{Rôles des membres du groupe}
Bascol Kevin \& Reynaud Nicolas ont pré-établi l'organisation.\\
\indent Laoussing Kévin, quant à lui, a réalisé le diagramme de gantt sur planner.\\
\indent Reynaud Nicolas a aussi regardé un exemple de rapport de réunion et l'a rédigé.\\

\subsection*{Résultats de la réunion}
Nous avons choisi de faire un diagramme de gantt car celui-ci était le plus représentatif d'une planification.\\

La planification a été faite en accord avec tout les membres mais également en prenant en compte que le mois de décembre serait certainement très chargé ( partiels etc ). Voila pourquoi la plupart des documents devraient être rendus avant décembre.\\

La répartition du travail quant à elle a été décidée en fonction du bon vouloir de chacun tout en répartissant également la charge de travail qui avait été évaluée pour chaque tache.\\

Le diagramme a été fait à l'aide du logiciel planner, celui étant léger et simple d'utilisation, nous n'avons pas poussé la recherche plus loin puisque cela correspondait à nos attentes.
Il permet également d'exporter le diagramme en version html afin de simplifier l'accès.\\
Kévin Laoussing, responsable du diagramme, nous a présenté la première version du diagramme, qui a été validée.\\

A noter que les différents cahiers des tests seront réalisés en parallèle des documents qui leurs sont associés ( document de spécification \& document de conception générale et détaillée ).\\

\subsection*{Aléa lors de la réunion}
La réunion a été interrompue pendant 7 min par M. Aubri Ugo. \\

\subsection*{Prochaine réunion}
La prochaine réunion est fixée au 21 Octobre 2014.\\

\subsection*{Clôture de la séance}
La séance a été levée à 16h00. \\

\end{document}
