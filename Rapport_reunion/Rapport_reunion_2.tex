\documentclass{article}
\usepackage[utf8]{inputenc}
\usepackage[T1]{fontenc}
\usepackage[francais]{babel}
\usepackage{hyperref}
\usepackage{parselines} 
\hypersetup{
  unicode=false, % non-Latin characters bookmarks
  pdftoolbar=false, % show Acrobat's toolbar?
  pdfmenubar=false, % show Acrobat's menu?
  pdffitwindow=true, % page fit to window when opened
  pdftitle={Rapport Genie logiciel}, % title
  pdfauthor={REYNAUD Nicolas}, % author
  pdfsubject={thèse}, % subject of the document
  pdfnewwindow=true, % links in new window
  colorlinks=true, % false: boxed links; true: colored links
  linkcolor=black, % color of internal links
  citecolor=green, % color of links to bibliography
  filecolor=magenta, % color of file links
  urlcolor=blue % color of external links
}

\usepackage{manyfoot}
\usepackage[stable]{footmisc}
\usepackage{geometry}
\usepackage{graphicx}
\usepackage{amssymb}

\geometry{hmargin=3.0cm,vmargin=1.5cm}

\title{Rapport Reunion \#2}
\author{Auteur : Nicolas \bsc{Reynaud}}
\date{1 Octobre 2014}

\begin{document}

\maketitle
\newpage

\renewcommand{\contentsname}{Sommaire}
\tableofcontents
\newpage

\section{Rapport reunion \#1}

Présent à cette réunion : Bascol Kevin, Laoussing Kévin, Reynaud Nicolas.

\subsection{Ordre du jours}
Faire la planification, le diagramme de gantt et vérifier le tout.\\

\subsection{Rôle des membres de l'équipe}

Bascol Kevin \& Reynaud Nicolas on pré-établi l'organisation.\\
\indent Laoussing Kévin quand a lui a réalisé le diagramme de gantt sur planner.\\
\indent Reynaud Nicolas : Regarder un exemple de rapport de réunion et le faire.\\

\subsection{Aléa lors de la réunion}

La réunion a été interrompu pendant 7 min par M. Aubri Ugo. \\

\subsection{Résultat de la réunion}

Nous avons choisi de faire un diagramme de gantt car celui ci était le plus représentatif d'une planification.

La planification a été fait en accord avec tout les membres mais également en prenant en compte que le mois de décembre serait certainement très chargé ( partielle etc ). Voila pourquoi la plupart des documents seront rendu avant décembre.\\

A noté que les différents cahier des tests seront réalisé en parallèles des document qui leurs sont associés ( document de spécification \& document de conception générale et détaillée ).\\

La répartition de travail quand à elle a été décidée en fonction du bon vouloir de chacun tout en répartissant également la charge de travail qui avait été évalué pour chaque taches.\\

Le Diagramme a été fait a l'aide du logiciel planner, celui étant lège et simple d'utilisation, nous n'avons pas poussé la recherche plus loin car celui ci correspondait à nos attentes.
Il permettait également d'exporter le diagramme en version html afin de simplifier l'accès.\\

Kévin Laoussing responsable du diagramme nous à présenté sa version numéro 1 u diagramme, qui a été validé.\\
Le diagramme a été réalisé à la suite de notre réunion.

\subsection{Prochaine réunion}

La prochaine réunion est fixée au 21 Octobre 2014.\\

\subsection{Cloture de la séance}

La séance s'est levée à 16h00. \\

\end{document}

