\documentclass[12pt,a4paper]{article}
\usepackage[utf8]{inputenc}
\usepackage[francais]{babel}
\usepackage[T1]{fontenc}
\usepackage{amsmath}
\usepackage{amsfonts}
\usepackage{amssymb}
\usepackage{geometry}
\usepackage{enumitem}
\usepackage{hyperref}

\geometry{hmargin=3.0cm,vmargin=1.5cm}
\parindent=0pt

\title{Rapport Reunion \#14}
\author{Auteur : Nicolas \bsc{REYNAUD}}
\date{8 janvier 2015}

	
\begin{document}
%--------------------------Page de présentation------------------------------%
\maketitle

\newpage

Présents à cette réunion : Bascol Kevin, Laoussing Kévin, Reynaud Nicolas.

%--------------------------Ordre du jour------------------------------%
\subsection*{Ordre du jour}
\begin{itemize}[label = $\blacktriangleright$]
\item Préparation de la réunion.
\item Finissions sur le projet.
\end{itemize}

%--------------------------Bilan du travail réalisé------------------------------%
\subsection*{Bilan du travail réalisé}

\begin{itemize}[label = $\blacktriangleright$]
\item Complétion du cahier des tests d'intégration et rendu.

\item Complétion du cahier de tests de recette et rendu.

\item Débuggage des fonctionnalités implémentées.

\item Mise en place des tests JUnit.

\item Préparation de la réunion avec relecture du code.

\item Rédaction du manuel utilisateur.

\item Rédaction du bilan.
\end{itemize}


%--------------------------Décision de la réunion------------------------------%
\subsection*{Décisions prises}

\begin{itemize}[label = $\blacktriangleright$] 
\item Arrêter de râper les carottes dans le sens de la longueur.
\item Tout faire pour devenir assez fort psychologiquement afin d'arriver à ne pas pleurer en épluchant des oignons.
\end{itemize}


\end{document}