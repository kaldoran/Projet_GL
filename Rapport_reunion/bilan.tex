\documentclass[12pt,a4paper]{article}
\usepackage[utf8]{inputenc}
\usepackage[francais]{babel}
\usepackage[T1]{fontenc}
\usepackage{amsmath}
\usepackage{amsfonts}
\usepackage{amssymb}
\usepackage{geometry}
\usepackage{enumitem}
\usepackage{hyperref}

\geometry{hmargin=3.0cm,vmargin=1.5cm}
\parindent=0pt

\title{Bilan projet KNS}
\author{Auteur : Kévin \bsc{BASCOL}}
\date{8 Janvier 2015}

	
\begin{document}
\maketitle

\newpage

\section*{Outils}

\subsection*{Diagrammes}
\begin{itemize}
\item Pour les diagrammes de Gantt nous avons utilisé le logiciel gratuit \textbf{planner}.
\item Pour les diagrammes de cas d'utilisation, de classe nous avons utilisé le site \textbf{genmymodel.com} permettant un travail collaboratif via internet.
\item genmymodel n'étant pas assez complet pour les diagrammes de sequence nous avons utilisé le site \textbf{lucidchart.com} permettant de travailler en même temps et en temps réel 
\end{itemize}

\subsection*{Gestionnaire de version}
Nous avons utilisé le gestionnaire de version \textbf{git}, en utilisant les services en ligne proposés par \textbf{Github}.

\subsection*{Communications}
\begin{itemize}
\item Nicolas a mis à disposition une partie de son site \textbf{ujm.eu5.org/gl}, pour pouvoir mettre la planification en ligne et donc permettre une meilleure visibilité de celle-ci.
\item Lorsque nous ne pouvions pas nous rencontrer en personne au moment des réunions nous avons utiliser le logiciel \textbf{Skype} pour avoir un moyen simple et efficace pour faire la réunion.
\item Nous avons crée \textbf{une conversation facebook} dédiée au projet pour qu'on soit un maximum en contact en dehors des cours et réunions.
\end{itemize}

\subsection*{Autre}
\begin{itemize}
\item Afin de se mettre d'accord sur une interface graphique alors que nous étions en vacances nous avons utilisé \textbf{flockdraw.com} qui est un site de dessin collaboratif.
\item Les document ont été réalisé à l'aide de \textbf{LaTeX}. Ceci qui nous a permis de ne pas avoir de problème de compatibilité mais aussi d'avoir des normes de présentations simplement.
\end{itemize}

\section*{Codage}
\begin{itemize}
\item Nous nous sommes mis d'accord sur l'utilisation de \textbf{l'IDE Netbeans} par tous les membres du groupe pour permettre l'uniformité des dossiers git et la possibilité de mieux s'entre-aider en cas de problème lié à l'IDE.
\item L'implémentation \textbf{MVC} du programme permet une standardisation de notre code.
\end{itemize}


\section*{Bibliographie}
\begin{itemize}
\item \underline{Design-pattern, tête la première}, Eric Freeman et Elisabeth Freeman
avec Kathy Sierra et Bert Bates
\end{itemize}

\end{document}
