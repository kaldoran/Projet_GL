\documentclass{article}
\usepackage[utf8]{inputenc}
\usepackage[T1]{fontenc}
\usepackage[francais]{babel}
\usepackage{hyperref}
\usepackage{parselines} 
\hypersetup{
  unicode=false, % non-Latin characters bookmarks
  pdftoolbar=false, % show Acrobat's toolbar?
  pdfmenubar=false, % show Acrobat's menu?
  pdffitwindow=true, % page fit to window when opened
  pdftitle={Rapport Genie Logiciel}, % title
  pdfauthor={REYNAUD Nicolas}, % author
  pdfsubject={report}, % subject of the document
  pdfnewwindow=true, % links in new window
  colorlinks=true, % false: boxed links; true: colored links
  linkcolor=black, % color of internal links
  citecolor=green, % color of links to bibliography
  filecolor=magenta, % color of file links
  urlcolor=blue % color of external links
}

\usepackage[stable]{footmisc}
\usepackage{geometry}
\usepackage{graphicx}
\usepackage{amssymb}

\geometry{hmargin=3.0cm,vmargin=1.5cm}

\title{Rapport Reunion \#1}
\author{Kevin \bsc{Bascol}, Kevin \bsc{Laoussing}, Nicolas \bsc{Reynaud}}
\date{07 Octobre 2014}

\begin{document}

\maketitle
\newpage

\renewcommand{\contentsname}{Sommaire}
\tableofcontents
\newpage


\section{Scénario du projet}

L'entreprise $\textbf{K}$ould$\textbf{N}$ot$\textbf{K}$onnect, en référence à leurs créateurs, désire la création d'un outil permettant de gérer et d'unifier leurs serveurs FTP.

Le logiciel aura pour nom KouldNotShare et sera développé en Java.

\section{Les outils}

Lors de la première réunion, nous avons décidé d'utiliser les outils suivant : 
\begin{itemize}
	\item Github : Site de gestion de version associé au gestionnaire en ligne de commande Git.
	\item Planner : Logiciel de planification ( Diagramme de Gantt ).
	\item Netbean : IDE de développement Java.
	\item Texmaker / Latex : Pour la rédaction des rapports.
\end{itemize}

\section{Mise en place des outils}

Lors de notre première réunion, nous avons également mis en place ( configuration et clonage du répertoire) de l'utilitaire Git. \\
Celle-ci a également été suivit d'une formation à Git ( Ligne de commandes ). \\

Nous en avons profité pour découvrir et tester Planner. \\
De plus, un modèle latex de compte rendu a été mis en place. \\

\section{Éclaircissement du sujet}

Nous avons cherché à approfondir le sujet, pour que chaque personne de l'équipe comprenne le sujet avant de nous lancer dans la planification, et nous avons réfléchi sur les fonctionnalités de notre sujet.\linebreak

Nous nous sommes mis d'accord sur le fait qu'il y aura un côté client et un côté serveur, tous deux développés par notre équipe.
\end{document}

