\documentclass{article}
\usepackage[utf8]{inputenc}
\usepackage[T1]{fontenc}
\usepackage[francais]{babel}
\usepackage{hyperref}
\usepackage{parselines} 
\hypersetup{
  unicode=false, % non-Latin characters bookmarks
  pdftoolbar=false, % show Acrobat's toolbar?
  pdfmenubar=false, % show Acrobat's menu?
  pdffitwindow=true, % page fit to window when opened
  pdftitle={Rapport Genie Logiciel}, % title
  pdfauthor={REYNAUD Nicolas}, % author
  pdfsubject={report}, % subject of the document
  pdfnewwindow=true, % links in new window
  colorlinks=true, % false: boxed links; true: colored links
  linkcolor=black, % color of internal links
  citecolor=green, % color of links to bibliography
  filecolor=magenta, % color of file links
  urlcolor=blue % color of external links
}

\usepackage[stable]{footmisc}
\usepackage{geometry}
\usepackage{graphicx}
\usepackage{amssymb}

\geometry{hmargin=3.0cm,vmargin=1.5cm}

\title{Rapport Reunion \#1}
\author{Kevin \bsc{Bascol}, Kevin \bsc{Laoussing}, Nicolas \bsc{Reynaud}}
\date{07 Octobre 2014}

\begin{document}

\maketitle
\newpage

\renewcommand{\contentsname}{Sommaire}
\tableofcontents
\newpage


\section{Introduction}

L'entreprise $\textbf{K}$ould$\textbf{N}$ot$\textbf{K}$onnect, en référence à leurs créateurs désire la création d'un outil permettant de gérer et unifier leurs serveurs FTP.

Logiciel ayant pour nom KouldNotShare, sera développé en Java.

\section{Les outils}

Lors de la première réunion, nous avons décidé d'utiliser les outils suivant : 
\begin{itemize}
	\item Github - Git : Site de gestion de version associé au gestionnaire en ligne de commande Git.
	\item Planner : Logiciel planification ( Diagramme de Gantt ).
	\item Netbean : IDE de développement Java.
	\item Texmaker / Latex : Pour la rédaction des rapports
\end{itemize}

\section{Mise en place des outils}

Lors de notre première réunion, nous avons également mis en place ( Configuration, et clonage du répertoire) de l'utilitaire Git. \\
Celle ci à également été suivit d'une formation à Git ( Ligne de commandes ). \\

Nous en avons profité pour découvrir / Tester Planner. \\
De plus, un modèle latex de compte rendu a été mis en place. \\

\section{Éclaircir le sujet}





\end{document}

