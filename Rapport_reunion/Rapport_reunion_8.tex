\documentclass[12pt,a4paper]{article}
\usepackage[utf8]{inputenc}
\usepackage[francais]{babel}
\usepackage[T1]{fontenc}
\usepackage{amsmath}
\usepackage{amsfonts}
\usepackage{amssymb}
\usepackage{geometry}
\usepackage{enumitem}
\usepackage{hyperref}
\usepackage{color}

\geometry{hmargin=3.0cm,vmargin=1.5cm}
\parindent=0pt

\title{Rapport Reunion \#8}
\author{Auteur : Nicolas \bsc{REYNAUD}}
\date{27 Novembre 2014}

	
\begin{document}
%--------------------------Page de présentation------------------------------%
\maketitle

\newpage

Présents à cette réunion : Laoussing Kévin, Reynaud Nicolas, Bascol Kevin.\\


%--------------------------Ordre du jour------------------------------%
\subsection*{Ordre du jour}
\begin{itemize}[label = $\ast$]
\item Diagramme de classe
\item Modification de la planification.
\item Détermination des fonctionnalités prioritaire.
\end{itemize}

%--------------------------Bilan du travail réalisé------------------------------%
\subsection*{Bilan du travail réalisé}

\begin{itemize}[label = $\ast$]
	\item Le diagramme de séquence à été réalisé par l'équipe.
	\item Changement du diagramme de planification par Kevin Laoussing.
\end{itemize}


%--------------------------Décision de la réunion------------------------------%
\subsection*{Décision de la réunion}

\begin{itemize}[label = $\ast$]
	\item Changement du site GenMyModel (Outil permettant la création de diagramme de séquence / classe etc)  pour \url{https://www.lucidchart.com}.
	\item https://www.lucidchart.com , proposé par Kévin Bascol nous laisse droit à 14 jours d'essai, ceci 	étant largement suffisant pour crée le diagramme de séquence.
	\item GenMyModel à été remplacé car celui ci ne permettait pas d'indiquer les actions asynchrones.
	\item Utilisation de lucidchart pour ça simplicité d'utilisation et sa capacité a travaillé en ligne a plusieurs.
	\item Déplacement de la période de codage pour la faire correspondre aux vacances, ce qui nous permettra d'être à notre plein potentiel.
	\item Raccourcissement du temps pour la rédaction du bilan, passant de 9 jours a 2 jours.
	\item La fonctionnalité de statistique sera réduite à sont minimum ( Pas de graphe, peut d'information ).
\end{itemize}

%--------------------------Prévision du travail à réaliser------------------------------%
\subsection*{Prévision du travail à réaliser}

\begin{itemize}[label = $\ast$]
	\item Finir la rédaction cahier de tests d'intégration.
\end{itemize}

%----------------------------------Prochaine réunion------------------------------------%
\subsection*{Prochaine réunion : 4 decembre 2014}

\textbf{Ordre du jour} : Début du code et état d'avancement.\\


\end{document}