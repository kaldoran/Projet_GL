\documentclass[12pt,a4paper]{article}
\usepackage[utf8]{inputenc}
\usepackage[francais]{babel}
\usepackage[T1]{fontenc}
\usepackage{amsmath}
\usepackage{amsfonts}
\usepackage{amssymb}
\usepackage{geometry}
\usepackage{enumitem}
\usepackage{hyperref}
\usepackage{color}

\geometry{hmargin=3.0cm,vmargin=1.5cm}
\parindent=0pt

\title{Rapport Reunion \#8}
\author{Auteur : Nicolas \bsc{REYNAUD}}
\date{27 Novembre 2014}

	
\begin{document}
%--------------------------Page de présentation------------------------------%
\maketitle

\newpage

Présents à cette réunion : Laoussing Kévin, Reynaud Nicolas, Bascol Kevin.\\


%--------------------------Ordre du jour------------------------------%
\subsection*{Ordre du jour}
\begin{itemize}[label = $\ast$]
\item Diagramme de classe.
\item Modification de la planification.
\item Détermination des fonctionnalités prioritaires.
\end{itemize}

%--------------------------Bilan du travail réalisé------------------------------%
\subsection*{Bilan du travail réalisé}

\begin{itemize}[label = $\ast$]
	\item Le diagramme de séquence à été réalisé par l'équipe.
	\item Changement du diagramme de Gantt par Kevin Laoussing.
\end{itemize}


%--------------------------Décision de la réunion------------------------------%
\subsection*{Décision de la réunion}

\begin{itemize}[label = $\ast$]
	\item Changement du site GenMyModel (Outil permettant la création de diagramme de séquence / classe etc)  pour \url{https://www.lucidchart.com}.
	\subitem https://www.lucidchart.com , proposé par Kevin Bascol, nous donne droit à 14 jours d'essai, ceci étant suffisant pour créer les diagrammes de séquence.
	\subitem GenMyModel à été remplacé car il ne permettait pas d'indiquer les appels asynchrones.
	\subitem Utilisation de lucidchart pour sa simplicité d'utilisation et la possibilité de collaborer en temps réel.
	\item Déplacement de la période de codage pour la faire correspondre aux vacances, ce qui nous permettra d'être à notre plein potentiel.
	\item Raccourcissement du temps pour la rédaction du bilan, passant de 9 jours a 2 jours.
	\item La fonctionnalité de statistique sera réduite à son minimum ( Pas de graphe, peu d'information ).
\end{itemize}

%--------------------------Prévision du travail à réaliser------------------------------%
\subsection*{Prévision du travail à réaliser}

\begin{itemize}[label = $\ast$]
	\item Finir la rédaction du document de conception détaillée.
	\item Finir la rédaction du cahier de tests d'intégration.
\end{itemize}

%----------------------------------Prochaine réunion------------------------------------%
\subsection*{Prochaine réunion : 4 décembre 2014}

\textbf{Ordre du jour} : Début du code et état d'avancement.\\


\end{document}