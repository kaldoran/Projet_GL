\documentclass[10pt,a4paper]{report}

\usepackage[utf8]{inputenc}
\usepackage{amsmath}
\usepackage{amsfonts}
\usepackage{amssymb}
\usepackage{graphicx}
\usepackage{color}
\usepackage{enumitem}
\usepackage[top=2cm, bottom=2cm, left=2cm, right=2cm]{geometry}

\usepackage{fancyhdr}
\pagestyle{fancy}

\fancyhead{}
\fancyfoot{} 
\lhead{\includegraphics{../Logo/logoKNKMini.jpg} \hspace{0.1cm} Kould Not Konect  \hspace{0.4cm} \vline}
\chead{Document de Spécification des Exigences}
\rhead{Kould Not Share}
\rfoot{\thepage}

\author{Kevin BASCOL, Kevin LAOUSSING, Nicolas REYNAUD}
\title{Document de spécification des exigences}
\date{3 Novembre 2014}

\makeatletter
\renewcommand{\thesection}{\@arabic\c@section}
\makeatother

\begin{document}

\makeatletter
	\begin{titlepage}
	
	\begin{figure}
		\begin{minipage}[c]{.46\linewidth}
		\end{minipage} \hfill
		\begin{minipage}[c]{.20\linewidth}
			\begin{center}
				\includegraphics{../Logo/logoKNK.jpg}\\
				{\large Kould Not Konect}
			\end{center}
		\end{minipage}
	\vspace{1cm}
	\end{figure}
	
	\centering
		{\Huge \textbf{\@title}}\\
		\hrule height 4pt
		\vspace{1.5cm}
		{\LARGE  Projet \textbf{Kould Not Share} v1.0}
		
		\vfill
		
		\@author\\
		\@date 
		\end{titlepage}
\makeatother
\setcounter{secnumdepth}{4}
\setcounter{tocdepth}{4}
\renewcommand{\contentsname}{Sommaire}
\tableofcontents
\thispagestyle{empty}
\setcounter{page}{0}
\newpage


\section{Introduction}

\subsection{Objectif du document}
Ce document présente les exigences logicielles et matérielles de la version 1.0 du projet Kould Not Share de l'entreprise Kould Not Konnect. Les responsables de ce projet sont Nicolas Reynaud, Kevin Laoussing et Kevin Bascol.

\textcolor{magenta}{\subsection{Portée du document}
Le logiciel de client/serveur FTP est un outil permettant à des particuliers ou des entreprises l'échange des données de manière contrôler. Par exemple, chaque utilisateurs propriétaires d'un fichier stocké dans le serveur FTP du logiciel, pourront paramétrer les droits de téléchargements sur ce fichier, et ainsi ils disposeront du pouvoir de restreindre ou d'augmenter l'accessibilité de son fichier...
}

\subsection{Définitions, acronymes et abréviations}
\begin{description}
\item[KNK] Kould Not Konect.
\item[KNS] Kould Not Share.
\item[FTP] File Transfer Protocol, protocole de communication destiné à l'échange informatique de fichiers sur un réseau TCP/IP.
\item[TCP/IP] Transmission Protocol/Internet Protocol, protocoles utilisés pour le transfert des données sur Internet.
\item[Protocole] Spécification de plusieurs règles pour un type de communication particulier.
\item[Client] Logiciel qui envoie des demandes à un serveur.
\item[Serveur] Dispositif informatique matériel ou logiciel qui offre des services, à différents clients.
\end{description}

\subsection{Références}

\textcolor{red}{\subsection{Vue d’ensemble}
[Cette section décrit le contenu du reste du document et explique comment le document est organisé.]}

\section{Description générale}

Ce projet consiste en la création d'un serveur FTP chez un particulier ou une entreprise, ayant accès à la machine sur laquelle est installé le programme. Les utilisateurs se verront donner l'accès par l'administrateur à certains dossiers de la machine sur laquelle est installé le serveur. Il pourront faire alors des échanges de fichiers dans ces répertoires à l'aide du client FTP.

\subsection{Perspectives du produit}

\textcolor{red}{\subsubsection{Interfaces système}
[Décrire les interfaces qui permettent la communication avec d’autres systèmes.]}

\textcolor{red}{\subsubsection{Interfaces utilisateurs}
[Décrire les interfaces utilisateur ou référer au prototype d’interface utilisateur.]}

\textcolor{red}{\subsubsection{Interfaces matérielles}
[Définir les interfaces entre le logiciel et le matériel, incluant la structure logique, les adresses physiques, le comportement attendu, etc.]}

\textcolor{red}{\subsubsection{Interfaces logicielles}
[Décrire les interfaces logicielles avec les autres composants du système. Ce peut être des composants achetés. Des composants de d’autres applications qui sont réutilisés ou des composants développés pour des sous-systèmes hors de la protée de la présente SEL et qui interagissent avec le présent système..]}

\textcolor{magenta}{\subsubsection{Interfaces de communication}
[Décrire les interfaces de communication avec les autres systèmes comme les réseaux locaux, les serveurs distants, etc.]}

\textcolor{red}{\subsubsection{Contraintes de mémoire}
[Indiquer les besoins en mémoire primaire et secondaire.]}

\subsection{Fonctions du produit}
\begin{itemize}
\item Permet de partager des dossiers entre un serveur et un ou plusieurs utilisateurs.
\item Permet de définir les droits d'accès aux dossiers du serveur.
\item Propose une interface claire et intuitive.
\end{itemize}

\subsection{Caractéristiques des utilisateurs}
Les utilisateurs de notre programme peuvent être des professionnels aguerris comme des utilisateurs néophytes. Il nous faudra donc proposer un programme sûr pour les entreprises mais aussi simple d'utilisation pour le particuliers.

\subsection{Contraintes}

\textcolor{red}{\subsection{Hypothèses et dépendances}
[Décrire tout élément de faisabilité technique, disponibilité de sous-système ou de composant ou toute autre hypothèse liée au projet de laquelle dépend la viabilité du logiciel.]}

\subsection{Exigences reportées}

\section{Exigences spécifiques}

\subsection{Fonctionnalités}

\subsubsection{Côté client}
\paragraph{Authentification}
	\begin{itemize}[label = $\triangleright$]
		\item Le programme devra permettra a un client de se connecter sur le client FTP.
		\item Le nom d'utilisateur et le mot de passe seront définit par l'administrateur du serveur. \textcolor{blue}{cf Authentification côté serveur}
		\item Le programme doit proposer un champ pseudo.
		\item Le programme doit proposer un champ mot de passe.
		\item Le programme doit permettre d'envoyer les données. \textcolor{blue}{cf Bouton envoyer}
		\item Le programme doit envoyé le mot de passe crypté au serveur. \textcolor{blue}{cf Protocoles}
		\item Le programme doit indiqué le succès a l'aide la phrase "Bonjour [pseudo]".
		\item Le programme doit indiqué si une erreur est survenu et le type d'erreur.
		\subitem Exemple avec la phrase suivante en cas de mauvais mot de passe "Erreur de mot de passe".
		\subitem Exemple en cas de pseudo non reconnu "L'utilisateur entré est invalide".
		\item Le programme doit permettre l'auto-connexion au démarrage du programme.
	\end{itemize}

\paragraph{Gestionnaire de marque-pages}
	\begin{itemize}[label = $\triangleright$]
		\item Le programme devrait proposer un gestionnaire de marque-pages de serveur.
		\item Le programme devrait afficher un bouton "Bookmarks"
		\item Cliquer sur le bouton "Bookmarks" devrait afficher la liste des marques-pages de l'utilisateur.
		\item Un marque-page doit se composer obligatoirement d'un nom, de l'adresse du serveur, de l'identifiant de l'utilisateur.
		\item Un marque-page doit avoir la possibilité d'enregistrer aussi le mot de passe de l'utilisateur si ce dernier le veut.
		\item Le programme devrait afficher un bouton "Mettre un marque-page".
		\item Cliquer sur le bouton "Mettre un marque-page" devrait afficher un champ de saisie et une case à cocher.
		\item Le champ de saisie devrait permettre à l'utilisateur d'enter le nom du marque-page.
		\item La case à cocher devrait permettre à l'utilisateur de choisir si il veut enregistre son mot de passe ou non.
	\end{itemize}
	
\paragraph{Limitations}
	\begin{itemize}[label = $\triangleright$]
		\item Le programme doit permettre à l'utilisateur de limiter la vitesse de téléchargement des fichiers.
		\item Pour cela le menu déroulant "Configuration" doit proposer un bouton "Limiter vitesse de téléchargement".
		\item Le programme doit permettre à l'utilisateur de donner une valeur précise à la limite de téléchargement.
		\item Cliquer sur le bouton "Limiter vitesse de téléchargement" doit faire apparaitre un curseur permettant de donner la valeur de la limite.
		\item Le programme doit permettre à l'utilisateur de limiter la vitesse de téléversement des fichiers.
		\item Pour cela le menu déroulant "Configuration" doit proposer un bouton "Limiter vitesse de téléversement".
		\item Le programme doit permettre à l'utilisateur de donner une valeur précise à la limite de téléversement.
		\item Cliquer sur le bouton "Limiter vitesse de téléversement" doit faire apparaitre un curseur permettant de donner la valeur de la limite.
		\item Le programme devrait permettre à l'utilisateur de bloquer le téléchargement de fichiers depuis sa machine pour certaines plages horaires.
		\item Pour cela le menu déroulant "Configuration" devrait proposer un bouton "Plages horaires téléchargement".
		\item Cliquer sur le bouton "Plages horaires téléchargement" doit faire apparaitre une liste d'horaires à cocher.
		\item Le programme devrait permettre à l'utilisateur de verrouiller le téléversement de fichiers depuis sa machine pour certaines plages horaires.
		\item Pour cela le menu déroulant "Configuration" devrait proposer un bouton "Plages horaires téléversement".
		\item Cliquer sur le bouton "Plages horaires téléversement" doit faire apparaitre une liste d'horaires à cocher.
	\end{itemize}
	
\paragraph{Utilisation des dossiers}
	\begin{itemize}[label = $\triangleright$]
		\item Le programme doit afficher l'arborescence des dossiers auquel l'utilisateur a accès sur le serveur courant.
		\item Le programme doit permettre à l'utilisateur de parcourir tous les sous-dossiers du répertoire qui lui a été fournit.
		\item Cliquer sur un dossier de l'arborescence doit afficher les sous-dossiers et fichiers de ce dossier.
		\item Le programme doit permettre à l'utilisateur de créer autant de sous-dossier qu'il veut dans le répertoire qui lui a été fourni.
		\item Faire un clic droit sur un dossier doit afficher un menu de gestion de dossier.
		\item Ce menu doit proposer un bouton "Nouveau sous-dossier".
		\item Cliquer sur le bouton "Nouveau sous-dossier" doit afficher un champ de saisie.
		\item Ce champ de saisie doit permettre à l'utilisateur d'entrer le nom du sous-dossier.
		\item Le programme doit permettre à l'utilisateur de supprimer n'importe quel sous-dossier du répertoire qui lui a été fourni.
		\item Le menu de gestion de dossier doit proposer un bouton "Supprimer le dossier".
		\item Cliquer sur le bouton "Supprimer le dossier" doit afficher une boite de confirmation de la suppression.
		\item Le programme doit permettre à l'utilisateur de créer des fichiers dans n'importe quel sous-dossier du répertoire qui lui a été fourni.
		\item Le menu de gestion de dossier doit proposer un bouton "Nouveau fichier vide".
		\item Cliquer sur le bouton "Nouveau fichier vide" doit créer un fichier sans extension dans le dossier sur lequel l'utilisateur a fait un clic droit.
		\item Le programme doit permettre à l'utilisateur de supprimer n'importe quel fichier dans les sous-dossiers du répertoire qui lui a été fourni.
		\item Faire un clic droit sur un fichier doit afficher un menu de gestion de fichier.
		\item Le menu de gestion de fichier doit proposer un bouton "Supprimer le fichier".
		\item Cliquer sur le bouton "Supprimer le fichier" doit afficher une boite de dialogue.
		\item Le programme doit supprimer le fichier si l'administrateur confirme la boite de dialogue.
	\end{itemize}
	
\paragraph{Communication réseau}

	\subparagraph{Procédure d'initialisation de la communication}

		\begin{itemize}[label = $\triangleright$]

			\item Le client doit se connecter sur le \textbf{port 21} du serveur initialiser une procédure d'échange de fichiers.

			\item Le client doit se connecter au port d'échange alloué par le serveur (que le serveur aura précédemment communiqué : cf exigence précédente) pour échanger des données.

			\item Le client doit résilier la connexion au \textbf{port 21} une fois qu'il a réussi à ce connecter au port d'échange.

			\item Le client doit utiliser un port libre c'est-à-dire que son numéro de port doit être supérieur à 1024.

			\item Un témoin graphique (de type diode) doit signaler la réussite de la connexion si la connexion est établie. 
			\end{itemize}

	\subparagraph{Procédure en cas d'échec d'initiation de la communication }

		\begin{itemize}[label = $\triangleright$]
			\item Le client doit prévenir l'utilisateur s'il ne parvient pas à établir une connexion avec le serveur sur le \textbf{port 21} à l'aide d'une fenêtre de dialogue affichant les messages suivant : 

			\begin{itemize}
				\item \textit{"Impossible d'établir une connexion avec le serveur $\textless nom\_du\_serveur\textgreater$:\\
-Vérifiez l'état de votre connexion Internet.\\
-Vérifiez que le nom du serveur n'est pas erroné."\\}
Si la source du problème provient du réseau (problème de routage, connexion coupée) ou si le serveur est inexistant.

				\item \textit{"Impossible d'établir une connexion avec le serveur $\textless nom\_du\_serveur\textgreater$: le serveur est surchargé."\\}
Si le nombre de connexion autorisé par le serveur est atteint.

				\item \textit{"Impossible d'établir une communication avec le serveur $\textless nom\_du\_serveur\textgreater$:\\
$\textless nom\_du\_serveur\textgreater$ vous a banni !"\\}
Si le client a été auparavant banni par le serveur. 
			\end{itemize}

			\item Le client doit prévenir l'utilisateur s'il ne parvient pas à établir une connexion avec le serveur sur le \textbf{port d'échange} à l'aide d'une fenêtre de dialogue affichant le message suivant : \\
\textit{"Impossible d'établir une connexion avec le serveur $\textless nom\_du\_serveur\textgreater$ sur le port $\textless numero\_du\_port\_echange\textgreater$ : Vérifiez vos règle de pare-feu."\\}

			\item Les fenêtre de dialogue doivent avoir un bouton "Fermer", positionné à leur angle inférieur droit, pour fermer celles-ci.

			\item Les fenêtre de dialogue doivent avoir un bouton "Ré-essayer", positionné à leur angle inférieur droit à gauche du bouton fermer, pour que le client puisse retenter une connexion avec le serveur avec les mêmes paramètres de communications que la tentative précédente.

		\end{itemize}
		
	\subparagraph{Procédure en cas de coupure de la communication}

		\begin{itemize}[label = $\triangleright$]
			\item Le client doit prévenir l'utilisateur si la communication avec le serveur est interrompue à l'aide d'une fenêtre de dialogue affichant le message suivant:\\
\textit{"Connexion interrompue avec le serveur $\textless nom\_du\_serveur\textgreater$."}

			\item La fenêtre de dialogue doit avoir un bouton "Fermer".

			\item Le client et le serveur doivent automatiquement fermer le port qui était dédié à la communication interrompue.
		\end{itemize}
		
	\subparagraph{Procédure de fermeture de la communication}

		\begin{itemize}[label = $\triangleright$]
			\item Le client doit \textbf{clôturer} la communication en cas d'\textbf{upload}. 
		\end{itemize}
		
\paragraph{Module de statistique}

	\subparagraph{Fonctionnalités}

	Ce module permettra à l'utilisateur de contrôler son activité sur le serveur FTP, et de contrôler les opérations faites sur ces fichiers par les autres utilisateurs.\\

	Voici les exigences :
		\begin{itemize}[label = $\triangleright$]

			\item Le module de statistique doit comptabiliser le nombre d'opération de "download" de chacun des fichiers de la propriété de l'utilisateur, par utilisateurs (ceux qui téléchargent les fichiers).

			\item Le module de statistique doit afficher un rapport en temps réel contenant: 

			\begin{itemize}
				\item Le fichier le plus télécharger.

				\item L'utilisateur qui "download" le plus ses fichiers.

				\item Un histogramme qui trace le nombre de téléchargement de chaque fichier de la propriété de l'utilisateur.
			\end{itemize} 
		\end{itemize}

	\subparagraph{Interface graphique du client}
		\begin{itemize}[label = $\triangleright$]

			\item La fenêtre doit contenir deux parties : une partie contenant les informations textuelles, et une partie graphique contenant l'histogramme.

			\item La partie textuelle doit être la partie supérieur de la fenêtre.

			\item La partie graphique doit être la partie inférieur de la fenêtre.

			\item La partie textuelle doit contenir un champ de texte non-éditable indiquant le fichier de la propriété de l'utilisateur qui a été le plus télécharger.

			\item La partie textuelle doit contenir un champ de texte non-éditable indiquant l'utilisateur ayant fait le plus d'opération de "download" sur les fichiers de la propriété de l'utilisateur (celui qui consulte le module de statistique).

			\item La partie graphique doit contenir l'histogramme qui trace le nombre téléchargement de chaque fichier (cf exigence module de statistique du client).

			\item L'histogramme doit tracer des barres verticales.

			\item L'histogramme doit avoir autant de barre que de fichiers (de la propriété de l'utilisateur).

			\item Chaque barre doit être entièrement visible sur l'histogramme.

			\item Chaque barre doit être de couleur différente.

			\item Chaque barre doit être de hauteur proportionnel au nombre de téléchargement (cf exigence module de statistique).

			\item Chaque barre doit être de largeur inversement proportionnel au nombre de fichiers (de la propriété de l'utilisateur) : plus l'utilisateur partage de fichiers plus les barres seront fine.

			\item Le nombre de fichier doit apparaître en ordonnée de l'histogramme.

			\item Les noms de fichier doit être correspondre à un numéro unique, ce numéro unique doit apparaître en abscisse de l'histogramme.

			\item l'histogramme doit avoir une légende qui informe la correspondance des noms de fichier avec leur numéro.
		\end{itemize}




\subsubsection{Côté serveur}
\paragraph{Authentification}
	\begin{itemize}[label = $\triangleright$]
		\item Le programme doit être capable de recevoir et détecter l'arrivée d'un pseudo et d'un mot de passe.
		\item Le programme doit vérifié les informations qu'il à reçu.
		\item Le programme doit indiqué au client le succès ou l'échec d'authentification.
		\item L'administrateur doit pouvoir se connecter au serveur à l'aide d'identifiants spécifiques.
		\item Le programme détecter que l'utilisateur est un administrateur et lui donner tout les droits.	
	\end{itemize}
	
\paragraph{Partage de fichiers}
	\begin{itemize}[label = $\triangleright$]
		\item Le programme doit partager les sous-dossier à partir de l'endroit où le programme à été lancé.
		\item Le programme doit pouvoir partager un fichier avec plusieurs personnes.
		\item \textcolor{red}{Le programme doit pouvoir bloquer l'accès à un sous-dossier d'un dossier partagé. }
		\item Les fichiers partagés seront stocké sur le serveur.
	\end{itemize}
	
\paragraph{Gestionnaire de comptes utilisateurs}
	\begin{itemize}[label = $\triangleright$]
		\item Le programme doit permettre a un administrateur de crée des comptes et de leurs attribuer des fichiers.
		\item Le programme doit permettre de crée des utilisateurs temporaires.
		\item Le programme doit se souvenir de tout les comptes crée.
		\item Le programme doit permettre d'attribuer des droits sur un fichier pour un compte donné.
		\item Le programme doit permettre de bannir une personne
	\end{itemize}

\paragraph{Bannissement d'un compte}
	\begin{itemize}[label = $\triangleright$]
		\item Le programme doit permettre de retirer les droits à une personne.
		\item Le programme doit stocker cette perte de droits.
		\item Le programme devrait permettre de bannir temporairement une personne.
		\item Le programme devrait pouvoir permettre de donner la raison du bannissement.
		\item Le programme devrait permettre un bannissement partiel d'un compte
	\end{itemize}
	
\paragraph{Bannissement temporaire}
	\begin{itemize}[label = $\triangleright$]
		\item Le programme devrait permettre de bannir un compte pour une durée déterminée.
		\item Le programme devrait pouvoir donner le temps de bannissement restant.
		\item Le programme devrait pouvoir associé une raison à se bannissement temporaire.
	\end{itemize}
	
\paragraph{Bannissement partiel}
	\begin{itemize}[label = $\triangleright$]
		\item Le programme devrait permettre de retirer les droits d'une personne uniquement sur une partie du serveur.
		\item Le programme devrait pouvoir stocker la ou les raison(s) de se bannissement partiel.
		\item Le programme devrait pouvoir associé une durée à se bannissement partiel.
	\end{itemize}
	
\paragraph{Stockage des comptes}
	\begin{itemize}[label = $\triangleright$]
		\item Le programme doit contenir dans un fichier les mots de passes sous une forme cryptée \textcolor{blue}{cf Cryptage}
		\item Le programme doit stocker les droits associé à un compte dans ce même fichier.
		\item Le programme doit stocker les comptes temporaires, les compte normaux, le compte administrateur et les comptes bannis dans 3 fichiers différents
		\item Toutes personnes qui n'est pas présent dans un des quatre fichiers est considérée comme inconnue.
	\end{itemize}
		
	\subparagraph{Compte Administrateur}
		\begin{itemize}[label = $\triangleright$]
			\item Le compte Administrateur doit être UNIQUE
			\item Le compte administrateur sera stocké dans un fichier à part
			\item Le fichiers qui stockera le compte administrateur n'aura pas d'extension
			\item Le nom du fichier qui stockera le compte administrateur aura pour nom .shadow \textcolor{blue}{cf linux}
			\item Le compte administrateur sera stocké sous la forme : pseudo@motDePasse:/CheminDeReference
			\item L'administrateur ne pourra distribuer les droits que à partir du Chemin de référence.
			\item L'administrateur aura priorité ABSOLUE sur tous les comptes.
			\item L'accès au compte administrateur doit se faire en accès direct à la machine. 
			\item AUCUN accès au compte administrateur pourra être fait a partir d'un client.
		\end{itemize}
		
	\subparagraph{Comptes normaux}
		\begin{itemize}[label = $\triangleright$]
			\item Le programme stockera les pseudos et les mots de passe dans un fichier .acc
			\item Le stockage sera sous la forme : pseudo@motDePasse:/fichier\#/autrefichier
		\end{itemize}
	\subparagraph{Comptes temporaires}
		\begin{itemize}[label = $\triangleright$]
			\item les comptes temporaires seront stockés dans un fichier .acctmp
			\item Le programme doit stocker le temps restant au compte.
			\item A la fin du temps le programme devra supprimer le compte lors de la prochaine tentative de connexion.
			\item Lors de la suppression le programme doit signaler a l'utilisateur que sont compte temporaire a été supprimé.
			\item Le stockage sera sous la forme : TempsEnSeconde|pseudo@motDePasse:/fichier\#/autrefichier
		\end{itemize}
	\subparagraph{Comptes bannis}
		\begin{itemize}[label = $\triangleright$]
			\item Le programme doit stocker les comptes bannis dans un fichier .accban
			\item Le programme devrait indiqué la raison du bannissement au client lors des tentatives de connexion.
			\item Le programme devrait pouvoir stocker le temps restant de bannissement.
			\item Le stockage sera sous la forme : TempsEnSeconde|pseudo@motDePasse:/fichier\#/autrefichier
			\item A la fin du TempsEnSeconde le compte devrait être réactivé.
		\end{itemize}
		

	
\paragraph{Limitations}
	\begin{itemize}[label = $\triangleright$]
		\item Le programme doit permettre à l'administrateur de limiter la vitesse de téléchargement des fichiers par les utilisateurs. 
		\item Pour cela le menu déroulant "Configuration" doit proposer un bouton "Limiter vitesse de téléchargement". 
		\item Le programme doit permettre à l'administrateur de donner une valeur précise à la limite de téléchargement. 
		\item Cliquer sur le bouton "Limiter vitesse de téléchargement" doit faire apparaitre un curseur permettant de donner la valeur de la limite.
		\item Le programme doit permettre à l'administrateur de limiter la vitesse de téléversement des fichiers par les utilisateurs.
		\item Pour cela le menu déroulant "Configuration" doit proposer un bouton "Limiter vitesse de téléversement".
		\item Le programme doit permettre à l'administrateur de donner une valeur précise à la limite de téléversement.
		\item Cliquer sur le bouton "Limiter vitesse de téléversement" doit faire apparaitre un curseur permettant de donner la valeur de la limite.
		\item Le programme devrait permettre à l'administrateur de bloquer le téléchargement de fichiers pour certaines plages horaires.
		\item Pour cela le menu déroulant "Configuration" devrait proposer un bouton "Plages horaires téléchargement".
		\item Cliquer sur le bouton "Plages horaires téléchargement" doit faire apparaitre une liste d'horaires à cocher.
		\item Le programme devrait permettre à l'administrateur de bloquer le téléversement de fichiers pour certaines plages horaires.
		\item Pour cela le menu déroulant "Configuration" devrait proposer un bouton "Plages horaires téléversement".
		\item Cliquer sur le bouton "Plages horaires téléversement" doit faire apparaitre une liste d'horaires à cocher.
		\item Le programme doit empêcher les utilisateurs de téléverser les fichiers ayant une extension interdite.
		\item Le menu déroulant "Configuration" doit proposer un bouton "Extensions interdites".
		\item Cliquer sur le bouton "Extensions interdites" doit afficher un champ de saisie.
		\item Le champ de saisie doit être prérempli avec les extensions déjà interdites.
		\item Un bouton "Valider" doit permettre de valider le contenu du champ de saisie.
	\end{itemize}
	
\paragraph{Utilisation des dossiers}
	\begin{itemize}[label = $\triangleright$]
		\item Le programme doit afficher l'arborescence des dossiers présents sur le serveur.
		\item Le programme doit permettre à l'administrateur de parcourir tous les dossiers sur le serveur.
		\item Cliquer sur un dossier de l'arborescence doit afficher les sous-dossiers et fichiers de ce dossier.
		\item Le programme doit permettre à l'administrateur de créer autant de dossier qu'il veut sur le serveur.
		\item Faire un clic droit sur un dossier doit afficher un menu de gestion de dossier.
		\item Ce menu doit proposer un bouton "Nouveau sous-dossier".
		\item Cliquer sur le bouton "Nouveau sous-dossier" doit afficher un champ de saisie.
		\item Ce champ de saisie doit permettre à l'administrateur d'entrer le nom du sous-dossier.
		\item Le programme doit permettre à l'administrateur de supprimer n'importe quel dossier sur le serveur.
		\item Le menu de gestion de dossier doit proposer un bouton "Supprimer le dossier".
		\item Cliquer sur le bouton "Supprimer le dossier" doit afficher une boite de dialogue demandant la confirmation de la suppression.
		\item Le programme doit permettre à l'administrateur de créer des fichiers dans n'importe quel dossier sur le serveur.
		\item Le menu de gestion de dossier doit proposer un bouton "Nouveau fichier vide".
		\item Cliquer sur le bouton "Nouveau fichier vide" doit créer un fichier sans extension dans le dossier sur lequel l'administrateur a fait un clic droit.
		\item Le programme doit permettre à l'administrateur de supprimer n'importe quel fichier sur le serveur.
		\item Faire un clic droit sur un fichier doit afficher un menu de gestion de fichier
		\item Le menu de gestion de fichier doit proposer un bouton "Supprimer le fichier".
		\item Cliquer sur le bouton "Supprimer le fichier" doit afficher une boite de dialogue.
		\item Le programme doit supprimer le fichier si l'administrateur confirme la boite de dialogue.
	\end{itemize}	
\paragraph{Communication réseau}

	\subparagraph{Procédure d'initialisation de la communication}

		\begin{itemize}[label = $\triangleright$]
			\item  Du démarrage jusqu'à sa fermeture, le serveur doit être à l'écoute sur le port \textbf{port 21}.

			\item Le serveur doit allouer \textbf{un port d'échange supérieur à 1024} avec le client à chaque requête d'initialisation.

			\item Le serveur doit communiquer au client le port d'échange qu'il a alloué (cf exigence précédente.

		\end{itemize}
		
\subparagraph{Procédure de fermeture de la communication Client-Serveur}

	\begin{itemize}[label = $\triangleright$]
		\item Le serveur doit \textbf{clôturer} la communication en cas de \textbf{download}. 
	\end{itemize}
	
\paragraph{Module de statistique \\ \\}

	Ce module permettra à l'administrateur de contrôler les activités sur son serveur, et 	de pouvoir par la suite de modifier la configuration de son serveur pour mieux adapter celui-ci au besoin des utilisateur.

	\subparagraph{Fonctionnalités}

		\begin{itemize}[label = $\triangleright$]
			\item Le module de statistique doit mesurer la fréquence de connexion des utilisateurs clients.

			\item Le module de statistique doit comptabiliser le nombre d'opération de "download" de chaque fichier stocké dans le serveur.

			\item Le module de statistique doit comptabiliser le nombre d'opération de "download" par utilisateur.

			\item Le module de statistique doit comptabiliser le nombre d'opération de "upload" par utilisateur.

			\item Le module de statistique doit comptabiliser le nombre de téléchargement par tranche horaire (2 heures).

			\item Les tranches horaire devrait être configurable;

			\item Le module de statistique doit pouvoir afficher en temps réel en plus d'éditer un rapport toute les 2 semaines contenant les statistiques suivante:

			\begin{itemize}
				\item le fichier le plus télécharger.

				\item l'utilisateur qui "upload" le plus de fichiers.

				\item l'utilisateur qui "download" le plus de fichiers.

				\item un histogramme H1 qui trace le nombre de téléchargement de chaque fichier stocké dans le serveur.

				\item un histogramme H2 qui trace le nombre d'opération de "download" et de "upload" de chaque utilisateur.

				\item un histogramme H3 qui trace la moyenne du nombre d'opération de "download" et de "upload" par tranche de 2 heures.
			\end{itemize}

			\item Le rapport en temps réel doit d'être accessible via un bouton sur l'interface graphique du programme.

			\item Le rapport édité par le module de statistique doit être un fichier de format html (le navigateur pourra interpréter le fichier et tracer les graphiques).
		\end{itemize}
	
	
	\subparagraph{Interface graphique du serveur}
		\begin{itemize}[label = $\triangleright$]
			\item La fenêtre doit contenir deux parties : une partie contenant les informations textuelles, et une partie graphique contenant l'histogramme.

			\item La partie textuelle doit être la partie supérieur de la fenêtre.

			\item La partie graphique doit être la partie inférieur de la fenêtre.

			\item La partie textuelle doit contenir un champ de texte non-éditable indiquant le fichier de la propriété de l'utilisateur qui a été le plus télécharger.

			\item La partie textuelle doit contenir un champ de texte non-éditable indiquant l'utilisateur ayant fait le plus d'opération de "download" sur les fichiers de la propriété de l'utilisateur (celui qui consulte le module de statistique).

			\item La partie textuelle doit contenir un champ de texte non-éditable indiquant l'utilisateur ayant fait le plus d'opération de "upload" sur les fichiers de la propriété de l'utilisateur (celui qui consulte le module de statistique).

			\item La partie graphique doit être diviser en trois sous-parties, une partie pour chaque histogramme (cf exigence module de statistique).

			\item L'histogramme H1 doit être positionné dans la partie inférieur de la partie graphique.

			\item L'histogramme H2 doit être positionné dans la partie supérieur gauche de la partie graphique.

			\item L'histogramme H3 doit être positionné dans la partie supérieur droit de la partie graphique.

			\item Les trois histogrammes doivent tracer des barres verticales.

			\item Chaque barre des 3 histogrammes doivent être entièrement visible.

			\item Chaque barre des 3 histogrammes doivent être de hauteur proportionnel à la valeur des ordonnées.

			\item Chaque barre des 3 histogrammes doivent être de largeur inversement proportionnel au nombre de barres.

			\item L'histogramme H1 doit avoir autant de barre que de fichiers (de la propriété de l'utilisateur).

			\item Chaque barre de l'histogramme H1 doit être de couleur différente.

			\item Le nombre de fichier doit apparaître en ordonnée de l'histogramme H1.

			\item Les noms de fichier doit être correspondre à un numéro unique, ce numéro unique doit apparaître en abscisse de l'histogramme H1.

			\item L'histogramme H1 doit avoir une légende qui informe la correspondance des noms de fichier avec leur numéro.

			\item L'histogramme H2 doit avoir autant de couples de barres que d'utilisateurs ayant effectuer une opération de "download" et "upload" (de la propriété de l'utilisateur).

			\item L'histogramme H2 doit tracer des couples de barres de couleurs différentes. La couleur représente l'information "upload" et "download".

			\item Le nombre d'opération de "upload" et de "download" doit apparaître dans la même ordonnée de l'histogramme H2.

			\item Les noms de chaque utilisateur (ayant effectuer une opération de "upload" et "download") doit être correspondre à un numéro unique, ce numéro unique doit apparaître en abscisse de l'histogramme H2.

			\item L'histogramme H2 doit avoir une légende qui informe la correspondance des noms de fichier avec leur numéro.

			\item L'histogramme H3 doit avoir autant de barre que de tranches horaires (24 divisé par la durée de la tranche horaire).

			\item Les barres de l'histogramme H3 doit être colorier avec une alternance de deux couleurs.

			\item Le nombre de fichier doit apparaître en ordonnée de l'histogramme H3.

			\item Les tranches horaires doivent apparaître en abscisse de l'histogramme H3.

		\end{itemize}

\subsubsection{Commun aux deux côtés}
\paragraph{Cryptage}
	\begin{itemize}[label = $\triangleright$]
		\item Le programme devrait proposer une fonction de cryptage de fichiers.
		\item Le menu de gestion de fichier devrait proposer un bouton "Cryptage du fichier"
		\item Le programme devrait enregistrer la clé de cryptage de chaque fichier crypté.
		\item Cliquer sur le bouton "Cryptage du fichier" devrait crypter le fichier et enregistrer la clé de cryptage.
		\item Le programme devrait permettre de décrypter les fichiers cryptés.
		\item Le programme devrait utiliser la clé de cryptage correspondante pour décrypter un fichier.
		\item Avant de télécharger un fichier crypté dont la clé de cryptage n'est pas enregistrée, le programme devrait afficher un champ de saisie.
		\item Ce champ de saisie devrait permettre à l'utilisateur d'entrer la clé de cryptage.
		\item Lors de l'envoie de la clé de cryptage, le programme devrait enregistrer celle-ci.
		\item \textcolor{blue}{Le programme devrait permettre de communiquer les clés de cryptage entre utilisateurs et administrateur.}
		\item Le programme doit crypter les mots de passe des utilisateurs
	\end{itemize}
	
\paragraph{Communication réseau}

	\begin{itemize}[label = $\triangleright$]
		\item Le client et le serveur doivent utiliser le \textbf{protocole FTP} pour échanger les données.
	\end{itemize}
	
\paragraph{Module statistique }

	\subparagraph{Interface graphique}
	
		\begin{itemize}[label = $\triangleright$]

			\item Le module statistique doit être accessible à partir d'un bouton "Statistique" dans la barre d'outil de la fenêtre principale.

			\item Le "clic" sur le bouton "Statistique" doit ouvrir une fenêtre (la fenêtre qui contiendra les informations du module statistique).

			\item Le "clic" sur le bouton "Statistique" doit initialiser les calculs statistiques (cf exigence module statistique" permettant d'avoir les informations statistiques au moment de l'événement "clic". Cela permet à l'utilisateur d'afficher les informations statistiques en temps réel.

			\item La fenêtre doit avoir un titre "Statistique".

			\item La fenêtre doit avoir une barre d'outil.
	
			\item La barre d'outil doit avoir un bouton "Exporter" positionner à son extrémité gauche.

			\item Le "clic" sur le bouton "Exporter" doit permettre la génération d'un rapport de statistique contenant les informations qu'affiche la fenêtre, \textbf{en priorité les graphiques}, dans un format html.

			\item La fenêtre doit avoir un bouton "Fermer" dont l'emplacement sera l'angle inférieur droit de la fenêtre. 

			item La barre
		\end{itemize}
	

\textcolor{red}{\subsection{Spécification des cas d’utilisation}
[Lorsqu’il y a une modélisation par cas d’utilisation, ceux-ci décrivent la majorité des exigences fonctionnelles du système ainsi que certaines exigences non-fonctionnelles. On peut référer au document de Spécification des cas d’utilisation.]}

\textcolor{red}{\subsection{Exigences supplémentaires}
[Décrire les exigences qui ne sont pas incluses dans les cas d’utilisation ainsi que les exigences non-fonctionnelles. On peut référer au document de Spécifications supplémentaires.]}

\textcolor{red}{\subsubsection{Utilisabilité}
[Décrire les exigences qui affectent l’utilisabilité comme, par exemple:\\
•	Le temps de formation nécessaire à un utilisateur normal ou expert avant d’être productif.\\
•	Les temps d’exécution pour les tâches courantes\\
•	Les exigences pour satisfaire aux standards d’utilisabilité d’interface graphique de, par exemple, Microsoft.]
\paragraph{\textless Nom de l’exigence d’utilisabilité 1 \textgreater}
[Description de l’exigence]}

\textcolor{red}{\subsubsection{Fiabilité}
[Décrire les exigences qui affectent la fiabilité comme, par exemple:\\
•	La disponibilité: le pourcentage d’heures d’utilisation, les périodes de maintenance, mode d’opération lors de dégradation, etc.\\
•	Durée moyenne de fonctionnement avant défaillance, exprimée en heures, en jours, en mois ou en années.\\
•	Durée moyenne de rétablissement, qui est le délai moyen de réparation d'une unité fonctionnelle après une défaillance.\\
•	Exactitude. précision, souvent définie par de normes, requise pour les extrants.\\
•	Nombre maximum d’anomalies exprimé habituellement en KLOC, en défaut par millier de ligne de code ou par points de fonction.\\
•	Criticité d’anomalie, mineure, significative, critique en décrivant ce que critique signifie.]
\paragraph{\textless Nom de l’exigence de fiabilité 1 \textgreater}
[Description de l’exigence]}

\textcolor{red}{\subsubsection{Performance}
[Décrire les caractéristiques de la performance du système. Référer les cas d’utilisation lorsque applicable.\\
•	Temps de réponse par transaction (moyen, maximum)\\
•	Débit (transactions par seconde)\\
•	Capacité (nombre de client ou de transaction que le système doit supporter)\\
•	Mode d’opération lors de dégradation (Mode d’opération acceptable lorsque la performance du système se détériore)\\
•	Utilisation de ressources (mémoire, disque, communications, etc.]
\paragraph{\textless Nom de l’exigence de performance 1 \textgreater}
[Description de l’exigence]}

\textcolor{red}{\subsubsection{Maintenabilité}
[Décrire les exigences qui permettent d’assurer le support et la maintenabilité du système comme, par exemple, les normes de codage, le conventions d’identification, le sbibliothèque3s de classe, l’accès à la maintenance, les services de maintenances, etc. 
\paragraph{\textless Nom de l’exigence de maintenance 1 \textgreater}
[Description de l’exigence]}


\section{Contraintes de conception}
\begin{description}
\item[Langage:] JAVA.
\item[Outil de developpement:] NetBean.
\end{description}
\textcolor{red}{\subsection{\textless Nom de la contrainte de conception 1 \textgreater}
[Description de l’exigence]}


\section{Sécurité}
Devront se faire de manière sécurisée:
\begin{itemize}
\item L'enregistrement des mots de passe des utilisateurs dans la base de donnée.
\item Les messages envoyés entre le serveur et chaque client.
\item L'enregistrement des clés de cryptage.
\end{itemize}



\section{Exigences de documentation utilisateur et d’aide en ligne}
Une documentation utilisateur complète doit être fournie avec le programme.


\textcolor{red}{\section{Normes applicables}
[Décrire par référence toutes normes applicables et les sections précises de ces normes qui s’appliquent au système. Cela inclut, par exemple, les normes de qualité, légales ou réglementaire, les normes industrielles d’utilisabilité, l’interopérabilité, les normes d’internationalisation, conformité au système d’exploitation, etc. ]}


\section{Classification des exigences fonctionnelles}
\bgroup
\def\arraystretch{1.5}
\begin{tabular}{|c|c|}
	\hline
	{\large \textbf{Fonctionnalité}} & {\large \textbf{Type}}\\
	\hline
	Utilisation des dossiers & Essentielle\\
	\hline
	Limitations & Souhaitable\\
	\hline
	Gestionnaire de marque-pages & Souhaitable\\
	\hline
	Cryptage & Optionnelle\\
	\hline
\end{tabular}
\egroup


\section{Annexes}

\end{document}