\documentclass[10pt,a4paper]{report}

\usepackage[utf8]{inputenc}
\usepackage{amsmath}
\usepackage{amsfonts}
\usepackage{amssymb}
\usepackage{graphicx}
\usepackage{color}
\usepackage{enumitem}
\usepackage[top=1cm, bottom=2cm, left=2cm, right=2cm]{geometry}

\usepackage{fancyhdr}
\pagestyle{fancy}

\fancyhead{}
\fancyfoot{} 
\lhead{\includegraphics{../Logo/logoKNKMini.jpg} \hspace{0.1cm} Kould Not Konect  \hspace{0.4cm} \vline}
\chead{Document de Spécification des Exigences}
\rhead{Kould Not Share}
\rfoot{\thepage}

\author{Kevin BASCOL, Kevin LAOUSSING, Nicolas REYNAUD}
\title{Document de spécification des exigences}
\date{3 Novembre 2014}

\makeatletter
\renewcommand{\thesection}{\@arabic\c@section}
\makeatother

\begin{document}

\makeatletter
	\begin{titlepage}
	
	\begin{figure}
		\begin{minipage}[c]{.46\linewidth}
		\end{minipage} \hfill
		\begin{minipage}[c]{.20\linewidth}
			\begin{center}
				\includegraphics{../Logo/logoKNK.jpg}\\
				{\large Kould Not Konect}
			\end{center}
		\end{minipage}
	\vspace{1cm}
	\end{figure}
	
	\centering
		{
		\hrule height 2pt
		\vspace{0.7cm}
		\Huge \textbf{\@title}}\\
		\vspace{0.7cm}
		\hrule height 2pt
		\vspace{1.5cm}
		{\LARGE  Projet \textbf{Kould Not Share} v1.0}
		
		\vfill
		
		\begin{tabular}{|c|c|c|}
			\hline
			Version & Date & Description\\
			\hline
			V.1 & 03/11/14 & Première version des exigences\\
			\hline
			 & & \\
			\hline
			 & & \\
			\hline
		\end{tabular}\\
		\vspace{1cm}
		\@author\\
		\end{titlepage}
\makeatother
\setcounter{secnumdepth}{5}
\setcounter{tocdepth}{5}
\renewcommand{\contentsname}{Sommaire}
\begingroup\makeatletter
\def\@makeschapterhead#1{%
  {\parindent \z@ \raggedright
    \normalfont
    \interlinepenalty\@M
    \Huge \bfseries  #1\par\nobreak
    \vskip 20pt% <---- à réduire pour avoir plus de place
  }}\makeatother
\tableofcontents
\endgroup
\thispagestyle{empty}
\setcounter{page}{0}
\newpage

\newgeometry{top=2cm, bottom=2cm, left=2cm, right=2cm}

\section{Introduction}

\subsection{Objectif du document}
Ce document présente les exigences logicielles et matérielles de la version 1.0 du projet Kould Not Share de l'entreprise Kould Not Konnect. Les responsables de ce projet sont Nicolas Reynaud, Kevin Laoussing et Kevin Bascol.

\subsection{Portée du document}
Le logiciel de client/serveur FTP est un outil permettant à des particuliers ou des entreprises l'échange des données de manière contrôler. Par exemple, chaque utilisateurs propriétaires d'un fichier stocké dans le serveur FTP du logiciel, pourront paramétrer les droits de téléchargements sur ce fichier, et ainsi ils disposeront du pouvoir de restreindre ou d'augmenter l'accessibilité de son fichier...


\subsection{Définitions, acronymes et abréviations}
\begin{description}
\item[KNK] Kould Not Konect.
\item[KNS] Kould Not Share.
\item[FTP] File Transfer Protocol, protocole de communication destiné à l'échange informatique de fichiers sur un réseau TCP/IP.
\item[TCP/IP] Transmission Protocol/Internet Protocol, protocoles utilisés pour le transfert des données sur Internet.
\item[Protocole] Spécification de plusieurs règles pour un type de communication particulier.
\item[Client] Logiciel qui envoie des demandes à un serveur.
\item[Serveur] Dispositif informatique matériel ou logiciel qui offre des services, à différents clients.
\item[Downloader] Anglicisme du mot "télécharger".
\item[Uploader] Anglicisme du mot "téléverser".
\end{description}


\section{Description générale}

Ce projet consiste en la création d'un serveur FTP chez un particulier ou une entreprise, ayant accès à la machine sur laquelle est installé le programme. Les utilisateurs se verront donner l'accès par l'administrateur à certains dossiers de la machine sur laquelle est installé le serveur. Il pourront faire alors des échanges de fichiers dans ces répertoires à l'aide du client FTP.\\

\subsection{Perspectives du produit}

\subsection{Fonctions du produit}
\begin{enumerate}
\item Permet de partager des dossiers entre un serveur et un ou plusieurs utilisateurs.
\item Permet de définir les droits d'accès aux dossiers du serveur.
\item Propose une interface claire et intuitive.
\end{enumerate}

\subsection{Caractéristiques des utilisateurs}
Les utilisateurs de notre programme peuvent être des professionnels aguerris comme des utilisateurs néophytes. Il nous faudra donc proposer un programme sûr pour les entreprises mais aussi simple d'utilisation pour le particuliers.

\section{Exigences spécifiques}

\subsection{Fonctionnalités}

\subsubsection{Côté client}
\paragraph{Authentification}

	\subparagraph{Interface graphique}\label{ClientIGA} 
	\begin{enumerate}
		\item Le formulaire de connexion doit être de type Pop-up.
		\item Le formulaire doit se situer au milieu de la fenêtre.
		\item Le Pop-Up de connexion doit apparaitre dès l'ouverture du logiciel.
		\item Quand le Formulaire sera présent le bouton déconnexion doit être désactive.
		\item Lorsqu'il est désactivé, le bouton déconnexion doit être grisé.
		\item Le titre de ce Pop-up doit être "Authentification".
		\item Le titre doit être centré.
		\item Le titre doit être suivi d'une barre ligne horizontale de couleur grise prenant au minimum 80\% du pop-up.
		\item Le formulaire doit avoir, à la suite de la barre horizontale trois champs permettant à l'utilisateur d'écrire.
		\item Les trois champs doivent être alignés avec les mots suivants : "Serveur", "Nom d'utilisateur" et "mot de passe" dans cette ordre.
		\item Un bouton centré, nommé "Se connecter" doit être présent en bas du pop-up.
		\item En cas d'erreur, un cadre aux bordures rouges doit apparaitre au dessus du champ concerné.
		\item Le reste de l'application (barre d'action non incluse) doit être grisé (voile gris au dessus) quand le pop-up est ouvert.
	\end{enumerate}
	
	\subparagraph{Fonctionnalité}
	\begin{enumerate}
		\item Le programme doit permettre à un client de se connecter sur le client FTP.
		\item Le nom d'utilisateur et le mot de passe doivent être définis par l'administrateur du serveur.
		\item Le programme doit proposer un champ de texte éditable pour entrer un pseudo.
		\item Le programme doit proposer un champ de texte éditable pour entrer mot de passe.
		\item Le programme doit permettre d'envoyer les données d'authentification via un protocole sécurisé.
		\item Le programme doit envoyer le mot de passe crypté au serveur.
		\item Le programme doit indiquer le succès de la connexion à l'aide la phrase "Bonjour [pseudo]" présente dans la barre des menus.
		\item Le programme doit indiquer si une erreur est survenue et son type.
		\item En cas d'erreur le programme doit indiquer l'erreur et redemander le mot de passe et/ou le login (en fonction de l'erreur survenue)
		\item En cas d'erreur sur l'adresse du serveur le programme doit indiquer l'erreur et redemander l'adresse du serveur.
		\item Le programme doit vérifier que l'adresse du serveur est une adresse IP V4.
		\item Le programme doit vérifier que le nom d'utilisateur ne dépasse pas 30 caractères.
		\item Le programme doit vérifier que le nom d'utilisateur n'a pas moins de 3 caractères.
		\item Le programme doit vérifier que le mot de passe comporte plus de 6 caractères.
		\item Le programme doit vérifier que le mot de passe comporte moins de 50 caractères.
		\begin{enumerate}
			\item Exemple avec la phrase suivante en cas de mauvais mot de passe "Erreur de mot de passe".
			\item Exemple en cas de pseudo non reconnu "L'utilisateur entré est invalide".
		\end{enumerate}
		\item Le programme doit permettre l'auto-connexion au démarrage du programme, pour ce faire l'utilisateur pourra cocher une case.
	\end{enumerate}

\paragraph{Gestionnaire de marque-pages}
	\begin{enumerate}
		\item Le programme devrait proposer un gestionnaire de marque-pages de serveur.
		\item Le programme devrait afficher un menu déroulant "Bookmarks" dans la barre des menus.
		\item Le menu déroulant "Bookmarks" devrait proposer un bouton "Afficher les marques-pages".
		\item Cliquer sur le bouton "Afficher les marques-pages" devrait afficher la liste des marques-pages de l'utilisateur dans une nouvelle page.
		\item Un marque-page doit se composer obligatoirement d'un nom, de l'adresse du serveur, de l'identifiant de l'utilisateur.
		\item Un marque-page doit avoir la possibilité d'enregistrer aussi le mot de passe de l'utilisateur si ce dernier le veut.
		\item Le menu déroulant "Bookmarks" devrait proposer un bouton "Mettre un marque-page".
		\item Cliquer sur le bouton "Mettre un marque-page" devrait afficher un champ de saisie et une case à cocher.
		\item Le champ de saisie devrait permettre à l'utilisateur d'enter le nom du marque-page.
		\item La case à cocher devrait permettre à l'utilisateur de choisir si il veut enregistre son mot de passe ou non.
	\end{enumerate}
	
\paragraph{Limitations}
	\begin{enumerate}
		\item Le programme doit proposer un menu déroulant "Configuration" dans la barre des menus.
		\item Le programme doit permettre à l'utilisateur de limiter la vitesse de téléchargement des fichiers.
		\item Pour cela le menu déroulant "Configuration" doit proposer un bouton "Limiter vitesse de téléchargement".
		\item Le programme doit permettre à l'utilisateur de donner une valeur précise à la limite de téléchargement.
		\item Cliquer sur le bouton "Limiter vitesse de téléchargement" doit faire apparaitre un curseur permettant de donner la valeur de la limite.
		\item Le programme doit permettre à l'utilisateur de limiter la vitesse de téléversement des fichiers.
		\item Pour cela le menu déroulant "Configuration" doit proposer un bouton "Limiter vitesse de téléversement".
		\item Le programme doit permettre à l'utilisateur de donner une valeur précise à la limite de téléversement.
		\item Cliquer sur le bouton "Limiter vitesse de téléversement" doit faire apparaitre un curseur permettant de donner la valeur de la limite.
		\item Le programme devrait permettre à l'utilisateur de bloquer le téléchargement de fichiers depuis sa machine pour certaines plages horaires.
		\item Pour cela le menu déroulant "Configuration" devrait proposer un bouton "Plages horaires téléchargement".
		\item Cliquer sur le bouton "Plages horaires téléchargement" doit faire apparaitre une liste d'horaires à cocher.
		\item Le programme devrait permettre à l'utilisateur de verrouiller le téléversement de fichiers depuis sa machine pour certaines plages horaires.
		\item Pour cela le menu déroulant "Configuration" devrait proposer un bouton "Plages horaires téléversement".
		\item Cliquer sur le bouton "Plages horaires téléversement" doit faire apparaitre une liste d'horaires à cocher.
	\end{enumerate}
	
\paragraph{Utilisation des dossiers}
	\begin{enumerate}
		\item Le programme doit afficher, dans la partie droite de la page principale, l'arborescence des dossiers auquel l'utilisateur a accès sur le serveur courant.
		\item Le programme doit permettre à l'utilisateur de parcourir tous les sous-dossiers du répertoire qui lui a été fournit.
		\item Cliquer sur un dossier de l'arborescence doit afficher les sous-dossiers et fichiers de ce dossier.
		\item Le programme doit permettre à l'utilisateur de créer autant de sous-dossier qu'il veut dans le répertoire qui lui a été fourni.
		\item Faire un clic droit sur un dossier doit afficher un menu de gestion de dossier.
		\item Ce menu doit proposer un bouton "Nouveau sous-dossier".
		\item Cliquer sur le bouton "Nouveau sous-dossier" doit afficher un champ de saisie.
		\item Ce champ de saisie doit permettre à l'utilisateur d'entrer le nom du sous-dossier.
		\item Le programme doit permettre à l'utilisateur de supprimer n'importe quel sous-dossier du répertoire qui lui a été fourni.
		\item Le menu de gestion de dossier doit proposer un bouton "Supprimer le dossier".
		\item Cliquer sur le bouton "Supprimer le dossier" doit afficher une boite de confirmation de la suppression.
		\begin{enumerate}[label=\arabic*.]
			\item La boite de confirmation de la suppression de dossier doit proposer deux boutons, "Valider" et "Annuler".
			\item Cliquer sur "Valider" doit supprimer le dossier puis fermer la boite de dialogue.
			\item Cliquer sur "Annuler" doit fermer la boite de dialogue.
		\end{enumerate}
		\item Le programme doit permettre à l'utilisateur de créer des fichiers dans n'importe quel sous-dossier du répertoire qui lui a été fourni.
		\item Le menu de gestion de dossier doit proposer un bouton "Nouveau fichier vide".
		\item Cliquer sur le bouton "Nouveau fichier vide" doit créer un fichier sans extension dans le dossier sur lequel l'utilisateur a fait un clic droit.
		\item Le programme doit permettre à l'utilisateur de supprimer n'importe quel fichier dans les sous-dossiers du répertoire qui lui a été fourni.
		\item Faire un clic droit sur un fichier doit afficher un menu de gestion de fichier.
		\item Le menu de gestion de fichier doit proposer un bouton "Supprimer le fichier".
		\item Cliquer sur le bouton "Supprimer le fichier" doit afficher une boite de dialogue.
		\begin{enumerate}[label=\arabic*.]
			\item La boite de confirmation de la suppression de fichier doit proposer deux boutons, "Valider" et "Annuler".
			\item Cliquer sur "Valider" doit supprimer le dossier puis fermer la boite de dialogue.
			\item Cliquer sur "Annuler" doit fermer la boite de dialogue.
		\end{enumerate}
	\end{enumerate}
	
\paragraph{Communication réseau}

	\subparagraph{Procédure d'initialisation de la communication}

		\begin{enumerate}

			\item Le client doit se connecter sur le \textbf{port 21} du serveur initialiser une procédure d'échange de fichiers.
			\item Le client doit se connecter au port d'échange alloué par le serveur (que le serveur aura précédemment communiqué : \textcolor{blue}{cf exigence 3.1.1.5.1}) pour échanger des données.
			\item Le client doit résilier la connexion au \textbf{port 21} une fois qu'il a réussi à ce connecter au port d'échange.
			\item Le client doit utiliser un port libre c'est-à-dire que son numéro de port doit être supérieur à 1024.
			\item Un témoin graphique (de type diode) doit signaler la réussite de la connexion si la connexion est établie. 
			\end{enumerate}

	\subparagraph{Procédure en cas d'échec d'initiation de la communication }

		\begin{enumerate}
			\item Le client doit prévenir l'utilisateur s'il ne parvient pas à établir une connexion avec le serveur sur le \textbf{port 21} à l'aide d'une fenêtre de dialogue affichant les messages suivant : 

			\begin{enumerate}[label=\arabic*.]
				\item \textit{"Impossible d'établir une connexion avec le serveur $< nom\_du\_serveur>$:\\
-Vérifiez l'état de votre connexion Internet.\\
-Vérifiez que le nom du serveur n'est pas erroné."\\}
Si la source du problème provient du réseau (problème de routage, connexion coupée) ou si le serveur est inexistant.

				\item \textit{"Impossible d'établir une connexion avec le serveur $< nom\_du\_serveur>$: le serveur est surchargé."\\}
Si le nombre de connexion autorisé par le serveur est atteint.

				\item \textit{"Impossible d'établir une communication avec le serveur $< nom\_du\_serveur>$:\\
$< nom\_du\_serveur>$ vous a banni !"\\}
Si le client a été auparavant banni par le serveur. 
			\end{enumerate}

			\item Le client doit prévenir l'utilisateur s'il ne parvient pas à établir une connexion avec le serveur sur le \textbf{port d'échange} à l'aide d'une fenêtre de dialogue affichant le message suivant : \\
\textit{"Impossible d'établir une connexion avec le serveur $< nom\_du\_serveur>$ sur le port $< numero\_du\_port\_echange>$ : Vérifiez vos règle de pare-feu."\\}

			\item Les fenêtre de dialogue doivent avoir un bouton "Fermer", positionné à leur angle inférieur droit, pour fermer celles-ci.

			\item Les fenêtre de dialogue doivent avoir un bouton "Ré-essayer", positionné à leur angle inférieur droit à gauche du bouton fermer, pour que le client puisse retenter une connexion avec le serveur avec les mêmes paramètres de communications que la tentative précédente.

		\end{enumerate}
		
	\subparagraph{Procédure en cas de coupure de la communication}

		\begin{enumerate}
			\item Le client doit prévenir l'utilisateur si la communication avec le serveur est interrompue à l'aide d'une fenêtre de dialogue affichant le message suivant:\\
\textit{"Connexion interrompue avec le serveur $< nom\_du\_serveur>$."}

			\item La fenêtre de dialogue doit avoir un bouton "Fermer".

			\item Le client et le serveur doivent automatiquement fermer le port qui était dédié à la communication interrompue.
		\end{enumerate}
		
	\subparagraph{Procédure de fermeture de la communication}

		\begin{enumerate}
			\item Le client doit \textbf{clôturer} la communication en cas d'\textbf{upload}. 
		\end{enumerate}
		
\paragraph{Download et Upload}

	\begin{enumerate}
		\item Le client doit pouvoir downloader des fichiers à partir de n'importe quel serveur FTP, et à condition que la connexion soit établis entre le celui-ci et le client.
		\item Le client doit pouvoir downloader des fichiers sur le serveur FTP KouldNotShare uniquement s'il est y autorisé.
		\item Le client doit pouvoir uploader des fichiers sur n'importe quel serveur FTP, et à condition que la connexion soit établis entre le celui-ci et le client.
		\item Le client doit pouvoir uploader des fichiers sur le serveur FTP KouldNotShare uniquement s'il est y autorisé.
		\item Le client doit prévenir l'utilisateur si le transfert de fichier n'est pas autorisé : une fenêtre devrait s'afficher avec le message suivant :\\
		 "Le transfert du \textless fichier\_cible\_a\_télécharger \textgreater n'est pas autorisé par le serveur \textless nom\_du\_serveur \textgreater ".
		\item Le client doit afficher une barre de progression (sur laquelle figure le pourcentage de progression) lorsqu'il transfert un fichier.
		\item Le client doit pouvoir crypter transfert de fichier. 
	\end{enumerate}
\paragraph{Module de statistique}

	\subparagraph{Fonctionnalités}

	Ce module permettra à l'utilisateur de contrôler son activité sur le serveur FTP, et de contrôler les opérations faites sur ces fichiers par les autres utilisateurs.\\

	Voici les exigences :
		\begin{enumerate}

			\item Le module de statistique doit comptabiliser le nombre de downloads de chacun des fichiers de la propriété de l'utilisateur, par utilisateurs (ceux qui téléchargent les fichiers).

			\item Le module de statistique doit afficher un rapport en temps réel contenant: 

			\begin{enumerate}
				\item Le fichier le plus télécharger.
				\item L'utilisateur qui download le plus ses fichiers.
				\item Un histogramme qui trace le nombre de téléchargement de chaque fichier de la propriété de l'utilisateur.
			\end{enumerate} 
		\end{enumerate}

	\subparagraph{Interface graphique du client}
		\begin{enumerate}

			\item La fenêtre doit contenir deux parties : une partie contenant les informations textuelles, et une partie graphique contenant l'histogramme.

			\item La partie textuelle doit être la partie supérieur de la fenêtre.

			\item La partie graphique doit être la partie inférieur de la fenêtre.

			\item La partie textuelle doit contenir un champ de texte non-éditable indiquant le fichier de la propriété de l'utilisateur qui a été le plus télécharger.

			\item La partie textuelle doit contenir un champ de texte non-éditable indiquant l'utilisateur ayant downloader des fichiers de la propriété de l'utilisateur (celui qui consulte le module de statistique).

			\item La partie graphique doit contenir l'histogramme qui trace le nombre téléchargement de chaque fichier (\textcolor{blue}{cf exigences 3.1.1.7}).

			\item L'histogramme doit tracer des barres verticales.

			\item L'histogramme doit avoir autant de barre que de fichiers (de la propriété de l'utilisateur).

			\item Chaque barre doit être entièrement visible sur l'histogramme.

			\item Chaque barre doit être de couleur différente.

			\item Chaque barre doit être de hauteur proportionnel au nombre de téléchargement (\textcolor{blue}{cf exigences 3.1.1.7}).

			\item Chaque barre doit être de largeur inversement proportionnel au nombre de fichiers (de la propriété de l'utilisateur) : plus l'utilisateur partage de fichiers plus les barres seront fine.

			\item Le nombre de fichier doit apparaître en ordonnée de l'histogramme.

			\item Les noms de fichier doit être correspondre à un numéro unique, ce numéro unique doit apparaître en abscisse de l'histogramme.

			\item l'histogramme doit avoir une légende qui informe la correspondance des noms de fichier avec leur numéro.
		\end{enumerate}




\subsubsection{Côté serveur}
\paragraph{Authentification}
	
	\subparagraph{Interface graphique}
		L'interface graphique de la connexion, du coté du serveur sera la même que celle coté client.\\
		A la différence près que le champs serveur et sont input associé ne sera pas présent.\\
		C.f Page \pageref{ClientIGA} : \ref{ClientIGA}
	\subparagraph{Fonctionnalité}
	\begin{enumerate}
		\item Le programme doit être capable de recevoir et détecter l'arrivée d'un pseudo et d'un mot de passe.
		\item Le programme doit vérifier les informations qu'il à reçu.
		\item Le programme doit indiquer au client le succès ou l'échec d'authentification.
		\item L'administrateur doit pouvoir se connecter au serveur à l'aide d'identifiants spécifiques.
		\item Le programme doit détecter que l'utilisateur est un administrateur et lui donner tous les droits.	
	\end{enumerate}
	
\paragraph{Partage de fichiers}
	\begin{enumerate}
		\item Le programme doit partager les sous-dossiers à partir de l'endroit où le programme a été lancé.
		\item Le programme doit pouvoir partager un fichier avec plusieurs personnes.
		\item Le programme doit pouvoir bloquer l'accès à un sous-dossier d'un dossier partagé.
		\item Les fichiers partagés doivent être stockés sur le serveur.
	\end{enumerate}
	
\paragraph{Gestionnaire de comptes utilisateurs}
	\begin{enumerate}
		\item Le programme doit permettre à un administrateur de créer des comptes et de leurs attribuer des fichiers.
		\item Le programme doit permettre de créer des utilisateurs temporaires.
		\item Le programme doit se souvenir de tous les comptes crées.\textcolor{blue}{ c.f \ref{cpt} }
		\item Le programme doit permettre d'attribuer des droits sur un fichier pour un compte donné.\textcolor{blue}{ c.f \ref{bdv}}	
		\item Le programme doit permettre de bannir une personne.\textcolor{blue}{ c.f \ref{bgr}}
		\item Le programme doit permettre de supprimer un compte.\textcolor{blue}{ c.f \ref{bgr}}
	\end{enumerate}
	
\subparagraph{Interface graphique}
	\begin{enumerate}
		\item La gestion de compte se fera a l'aide de cliques dans la zone gauche du programme.\label{cpt} 
		\item Dans cette zone gauche se trouverons la liste des personnes ayant un accès au FTP.
		\item Un indicateur de type voyant, se trouvera a droite de chaque nom de compte, indiquant la présence ou non d'une personne en ligne.
			\begin{enumerate}[label=\arabic*.]
				\item Si la personne n'est pas connectée, le voyant sera rouge.
				\item Si la personne est connectée, le voyant sera vert.
			\end{enumerate} 
		\item Deux boutons seront présent en bas de cette zone.
			\begin{enumerate}[label=\arabic*.]
				\item Un bouton en bas a droite avec un "-" rouge dedans.\label{bgr} 
				\item Au clic sur le bouton "-" un pop-up s'ouvrira. Voir \ref{del} Page : \pageref{del}
				\item Un bouton en bas a gauche avec un "+" vert dedans.\label{bdv}
				\item Au clic sur le bouton "+" un pop-up s'ouvrira. Voir \ref{add} Page : \pageref{add}
			\end{enumerate} 
		\item Les comptes temporaires seront souligné dans la zone gauche.
		\item Le compte Administrateur ne sera pas présent dans cette liste.
		\item Les utilisateurs bannis seront en rouge.
		\item Les comptes bannis temporairement seront en gris clair.
		\item Les comptes bannis partiellement seront en orange.
		\item Les comptes normaux seront en noir.
		\item Lors du clic sur un compte, la fenêtre de droite changera, elle contiendra alors les répertoire auquel l'utilisateur sélectionné à accès. Voir \ref{chg} Page : \pageref{chg}
	\end{enumerate}
	
\paragraph{Bannissement d'un compte}
	\begin{enumerate}
		\item Le programme doit permettre de retirer les droits à une personne.
		\item Le programme doit stocker cette perte de droits.
		\item Le programme devrait permettre de bannir temporairement une personne.
		\item Le programme devrait pouvoir permettre de donner la raison du bannissement.
		\item Le programme devrait permettre un bannissement partiel d'un compte.
	\end{enumerate}
	
\paragraph{Bannissement temporaire}
	\begin{enumerate}
		\item Le programme devrait permettre de bannir un compte pour une durée déterminée.
		\item Le programme devrait pouvoir donner le temps de bannissement restant.
		\item Le programme devrait pouvoir associer une raison à ce bannissement temporaire.
	\end{enumerate}
	
\paragraph{Bannissement partiel}
	\begin{enumerate}
		\item Le programme devrait permettre de retirer les droits d'une personne uniquement sur une partie du serveur.
		\item Le programme devrait pouvoir stocker la ou les raison(s) de ce bannissement partiel.
		\item Le programme devrait pouvoir associer une durée à ce bannissement partiel.
	\end{enumerate}
	
\subparagraph{Interface graphique bannissement \& suppression}\label{del} 
	\begin{enumerate}

		\item Le pop-up doit contenir deux cases à cocher :
			\begin{enumerate}[label=\arabic*.]
				\item Une case alignée avec le texte "Supprimer le compte"
				\item Une case alignée avec le texte "Bannir le compte"
			\end{enumerate} 
		\item Seule une des deux cases doit pouvoir être cochée.
		\item En bas de ce pop-up doit se trouver deux boutons :
			\begin{enumerate}[label=\arabic*.]
				\item Un bouton "annuler", qui annulera toute la procédure.
				\item Un bouton "valider", qui validera la procédure.
			\end{enumerate} 
		\item Si la case suppression du compte est cochée aucun élément n'est ajouté.
		\item Si la case "Bannir le compte" est cochée alors les champs suivants seront ajouté en dessous de la case :
			\begin{enumerate}[label=\arabic*.]
				\item Un champs Motif sera ajouté associé au texte "Motif du bannissement"
				\item Une case à cochée sera présente associée au texte "Bannissement temporaire"
			\end{enumerate} 
		\item Si la case bannissement temporaire est cochée, un champs de type date sera ajouté en dessous. 
		\item Le programme devrait pouvoir stocker la ou les raison(s) de se bannissement partiel.
		\item Le programme devrait pouvoir associé une durée à se bannissement partiel.
	\end{enumerate}
	
	\subparagraph{Interface graphique ajout de compte}\label{add} 
		\begin{enumerate}
			\item Le pop-up doit contenir quatre champs :
			\begin{enumerate}[label=\arabic*.]
				\item Un champs de type texte associé au texte : "Nom d'utilisateur"
				\item Un champs de type texte associé au texte : "Mot de passe"
				\item Un champs de type sélecteur de fichier. Celui ci permettant d'indiquer, en les sélectionnant, les fichiers accessible par l'utilisateur.
				\item Un champs de type case à coché associé au texte : "Utilisateur temporaire".
			\end{enumerate}
			\item Si la case "Utilisateur temporaire" est cochée, un champs de type date s'ajoutera.
			\item En bas a droite de ce pop-up un bouton annulé.
			\item En bas a gauche de ce pop-up un bouton validé qui aura pour but d'ajouter l'utilisateur.
		\end{enumerate}
		
	\subparagraph{Interface graphique changement droits}\label{chg} 
		\begin{enumerate}
			\item Dans la partie gauche du programme ce trouvera par défaut la liste de tout les fichiers accessible sur le serveur.
			\item En cliquant sur un utilisateur seul a liste des fichiers accessible par l'utilisateur sera présente.
			\item En bas de cette fenêtre seront présent 2 champs
				\begin{enumerate}[label=\arabic*.]
					\item En bas à droite un "+" de couleur verte.
					\item A droite du bouton "+" un bouton "-" de couleur rouge.
				\end{enumerate}
			\item Il suffira alors de cliquer sur un fichier puis sur moins pour retirer les droits a l'utilisateur sélectionne sur le fichier sélectionne.
			\item En cliquant sur le "+" un pop-up apparaitra permettant de sélectionne les fichiers sur lesquels l'utilisateur gagnera les droits.
		\end{enumerate}
	
	\subparagraph{Stockage des comptes}
		\begin{enumerate}
			\item Le programme doit contenir dans un fichier les mots de passe sous une forme cryptée.
			\item Le programme doit stocker les droits associés à un compte dans ce même fichier.
			\item Le programme doit stocker les comptes temporaires, les comptes normaux, le compte administrateur et les comptes bannis dans 3 fichiers différents
			\item Toute personne qui n'est pas présente dans un des quatre fichiers est considérée comme inconnue.
		\end{enumerate}

		
	\subparagraph{Compte Administrateur}
		\begin{enumerate}
			\item Le compte Administrateur doit être UNIQUE.
			\item Le compte administrateur doit être stocké dans un fichier à part.
			\item Le fichiers qui stockera le compte administrateur ne doit pas avoir pas d'extension.
			\item Le nom du fichier qui stockera le compte administrateur doit avoir pour nom ".shadow".
			\item Le compte administrateur doit être stocké sous la forme : "pseudo@motDePasse:/CheminDeReference".
			\item L'administrateur ne doit pouvoir distribuer les droits que à partir du chemin de référence.
			\item L'administrateur doit avoir priorité ABSOLUE sur tous les comptes.
			\item L'accès au compte administrateur doit se faire en accès direct à la machine. 
			\item AUCUN accès au compte administrateur doit pouvoir être fait à partir d'un client.
		\end{enumerate}
		
	\subparagraph{Comptes normaux}
		\begin{enumerate}
			\item Le programme doit stocker les pseudos et les mots de passe dans un fichier ".acc".
			\item Le stockage doit être sous la forme : "pseudo@motDePasse:/fichier\#/autrefichier".
		\end{enumerate}
	\subparagraph{Comptes temporaires}
		\begin{enumerate}
			\item les comptes temporaires doivent être stockés dans un fichier ".acctmp".
			\item Le programme doit stocker le temps restant au compte.
			\item A la fin du temps, le programme doit supprimer le compte lors de la prochaine tentative de connexion.
			\item Lors de la suppression, le programme doit signaler à l'utilisateur que son compte temporaire a été supprimé.
			\item Le stockage sera sous la forme : "DateFinEnSeconde$ \vert $pseudo@motDePasse:/fichier\#/autrefichier".

		\end{enumerate}
	\subparagraph{Comptes bannis}
		\begin{enumerate}
			\item Le programme doit stocker les comptes bannis dans un fichier ".accban".
			\item Le programme devrait indiquer la raison du bannissement au client lors des tentatives de connexion.
			\item Le programme devrait pouvoir stocker le temps restant de bannissement.
			\item Le stockage sera sous la forme : "TimestampDeFin$ \vert $pseudo@motDePasse:/fichier\#/autrefichier".
			\item A la fin du Timestamp compte devrait être réactivé.
		\end{enumerate}
		
\paragraph{Limitations}
	\begin{enumerate}
		\item Le programme doit proposer un menu déroulant "Configuration" dans la barre des menus.
		\item Le programme doit permettre à l'administrateur de limiter la vitesse de téléchargement des fichiers par les utilisateurs. 
		\item Pour cela le menu déroulant "Configuration" doit proposer un bouton "Limiter vitesse de téléchargement". 
		\item Le programme doit permettre à l'administrateur de donner une valeur précise à la limite de téléchargement. 
		\item Cliquer sur le bouton "Limiter vitesse de téléchargement" doit faire apparaitre un curseur permettant de donner la valeur de la limite.
		\item Le programme doit permettre à l'administrateur de limiter la vitesse de téléversement des fichiers par les utilisateurs.
		\item Pour cela le menu déroulant "Configuration" doit proposer un bouton "Limiter vitesse de téléversement".
		\item Le programme doit permettre à l'administrateur de donner une valeur précise à la limite de téléversement.
		\item Cliquer sur le bouton "Limiter vitesse de téléversement" doit faire apparaitre un curseur permettant de donner la valeur de la limite.
		\item Le programme devrait permettre à l'administrateur de bloquer le téléchargement de fichiers pour certaines plages horaires.
		\item Pour cela le menu déroulant "Configuration" devrait proposer un bouton "Plages horaires téléchargement".
		\item Cliquer sur le bouton "Plages horaires téléchargement" doit faire apparaitre une liste d'horaires à cocher.
		\item Le programme devrait permettre à l'administrateur de bloquer le téléversement de fichiers pour certaines plages horaires.
		\item Pour cela le menu déroulant "Configuration" devrait proposer un bouton "Plages horaires téléversement".
		\item Cliquer sur le bouton "Plages horaires téléversement" doit faire apparaitre une liste d'horaires à cocher.
		\item Le programme doit empêcher les utilisateurs de téléverser les fichiers ayant une extension interdite.
		\item Le menu déroulant "Configuration" doit proposer un bouton "Extensions interdites".
		\item Cliquer sur le bouton "Extensions interdites" doit afficher un champ de saisie.
		\item Le champ de saisie doit être prérempli avec les extensions déjà interdites.
		\item Un bouton "Valider" doit permettre de valider le contenu du champ de saisie.
	\end{enumerate}
	
\paragraph{Utilisation des dossiers}
	\begin{enumerate}
		\item Le programme doit afficher l'arborescence des dossiers présents sur le serveur.
		\item Le programme doit permettre à l'administrateur de parcourir tous les dossiers sur le serveur.
		\item Cliquer sur un dossier de l'arborescence doit afficher les sous-dossiers et fichiers de ce dossier.
		\item Le programme doit permettre à l'administrateur de créer autant de dossier qu'il veut sur le serveur.
		\item Faire un clic droit sur un dossier doit afficher un menu de gestion de dossier.
		\item Ce menu doit proposer un bouton "Nouveau sous-dossier".
		\item Cliquer sur le bouton "Nouveau sous-dossier" doit afficher un champ de saisie.
		\item Ce champ de saisie doit permettre à l'administrateur d'entrer le nom du sous-dossier.
		\item Le programme doit permettre à l'administrateur de supprimer n'importe quel dossier sur le serveur.
		\item Le menu de gestion de dossier doit proposer un bouton "Supprimer le dossier".
		\item Cliquer sur le bouton "Supprimer le dossier" doit afficher une boite de dialogue demandant la confirmation de la suppression.
		\begin{enumerate}[label=\arabic*.]
			\item La boite de confirmation de la suppression de dossier doit proposer deux boutons, "Valider" et "Annuler".
			\item Cliquer sur "Valider" doit supprimer le dossier puis fermer la boite de dialogue.
			\item Cliquer sur "Annuler" doit fermer la boite de dialogue.
		\end{enumerate}
		\item Le programme doit permettre à l'administrateur de créer des fichiers dans n'importe quel dossier sur le serveur.
		\item Le menu de gestion de dossier doit proposer un bouton "Nouveau fichier vide".
		\item Cliquer sur le bouton "Nouveau fichier vide" doit créer un fichier sans extension dans le dossier sur lequel l'administrateur a fait un clic droit.
		\item Le programme doit permettre à l'administrateur de supprimer n'importe quel fichier sur le serveur.
		\item Faire un clic droit sur un fichier doit afficher un menu de gestion de fichier
		\item Le menu de gestion de fichier doit proposer un bouton "Supprimer le fichier".
		\item Cliquer sur le bouton "Supprimer le fichier" doit afficher une boite de dialogue.
		\begin{enumerate}[label=\arabic*.]
			\item La boite de confirmation de la suppression de fichier doit proposer deux boutons, "Valider" et "Annuler".
			\item Cliquer sur "Valider" doit supprimer le dossier puis fermer la boite de dialogue.
			\item Cliquer sur "Annuler" doit fermer la boite de dialogue.
		\end{enumerate}
	\end{enumerate}	
	
\paragraph{Communication réseau}

	\subparagraph{Procédure d'initialisation de la communication}

		\begin{enumerate}
			\item  Du démarrage jusqu'à sa fermeture, le serveur doit être à l'écoute sur le port \textbf{port 21}.

			\item Le serveur doit allouer \textbf{un port d'échange supérieur à 1024} avec le client à chaque requête d'initialisation.

			\item Le serveur doit communiquer au client le port d'échange qu'il a alloué (\textcolor{blue}{cf exigences 3.1.2.10.2}).

		\end{enumerate}
		
\subparagraph{Procédure de fermeture de la communication Client-Serveur}

	\begin{enumerate}
		\item Le serveur doit \textbf{clôturer} la communication en cas de \textbf{download}. 
	\end{enumerate}

\paragraph{Download et Upload}
	
	\begin{enumerate}
		\item Le serveur doit vérifier la légitimité de toutes les requêtes de types download et upload :
		\begin{enumerate}
			\item Si le client est authentifié et le fichier qu'il souhaite downloader est autorisé par son propriétaire.
			\item Si le client est authentifié et le fichier qu'il souhaite downloader est publique (n'importe qu'elle client est autorisé à le télécharger).
		\end{enumerate}
	\end{enumerate}
	
\paragraph{Module de statistique}
	\subparagraph*{ }
	Ce module permettra à l'administrateur de contrôler les activités sur son serveur, et 	de pouvoir par la suite de modifier la configuration de son serveur pour mieux adapter celui-ci au besoin des utilisateur.

	\subparagraph{Fonctionnalités}

		\begin{enumerate}
			\item Le module de statistique doit mesurer la fréquence de connexion des utilisateurs clients.

			\item Le module de statistique doit comptabiliser le nombre de downloads de chaque fichier stocké dans le serveur.

			\item Le module de statistique doit comptabiliser le nombre de downloads par utilisateur.

			\item Le module de statistique doit comptabiliser le nombre de uploads par utilisateur.

			\item Le module de statistique doit comptabiliser le nombre de téléchargement par tranche horaire (2 heures).

			\item Les tranches horaire devrait être configurable;

			\item Le module de statistique doit pouvoir afficher en temps réel en plus d'éditer un rapport toute les 2 semaines contenant les statistiques suivante:

			\begin{enumerate}[label=\arabic*.]
				\item le fichier le plus télécharger.

				\item l'utilisateur qui upload le plus de fichiers.

				\item l'utilisateur qui download le plus de fichiers.

				\item un histogramme H1 qui trace le nombre de téléchargement de chaque fichier stocké dans le serveur.

				\item un histogramme H2 qui trace le nombre de downloads et de uploads de chaque utilisateur.

				\item un histogramme H3 qui trace la moyenne du nombre de downloads et de uploads par tranche de 2 heures.
			\end{enumerate}

			\item Le rapport en temps réel doit d'être accessible via un bouton sur l'interface graphique du programme.

			\item Le rapport édité par le module de statistique doit être un fichier de format html (le navigateur pourra interpréter le fichier et tracer les graphiques).
		\end{enumerate}
	
	
	\subparagraph{Interface graphique}
		\begin{enumerate}
			\item La fenêtre doit contenir deux parties : une partie contenant les informations textuelles, et une partie graphique contenant l'histogramme.
			\item La partie textuelle doit être la partie supérieur de la fenêtre.
			\item La partie graphique doit être la partie inférieur de la fenêtre.
			\item La partie textuelle doit contenir un champ de texte non-éditable indiquant le fichier de la propriété de l'utilisateur qui a été le plus télécharger.
			\item La partie textuelle doit contenir un champ de texte non-éditable indiquant l'utilisateur ayant fait le plus de downloads sur les fichiers de la propriété de l'utilisateur (celui qui consulte le module de statistique).
			\item La partie textuelle doit contenir un champ de texte non-éditable indiquant l'utilisateur ayant fait le plus de upload sur les fichiers de la propriété de l'utilisateur (celui qui consulte le module de statistique).
			\item La partie graphique doit être diviser en trois sous-parties, une partie pour chaque histogramme (\textcolor{blue}{cf exigences 3.1.2.12}).
			\item L'histogramme H1 doit être positionné dans la partie inférieur de la partie graphique.
			\item L'histogramme H2 doit être positionné dans la partie supérieur gauche de la partie graphique.
			\item L'histogramme H3 doit être positionné dans la partie supérieur droit de la partie graphique.
			\item Les trois histogrammes doivent tracer des barres verticales.
			\item Chaque barre des 3 histogrammes doit être entièrement visible.
			\item Chaque barre des 3 histogrammes doit être de hauteur proportionnel à la valeur des ordonnées.
			\item Chaque barre des 3 histogrammes doit être de largeur inversement proportionnel au nombre de barres.

			\item L'histogramme H1 doit avoir autant de barre que de fichiers (de la propriété de l'utilisateur).
			\item Chaque barre de l'histogramme H1 doit être de couleur différente.
			\item Le nombre de fichier doit apparaître en ordonnée de l'histogramme H1.
			\item Les noms de fichier doit être correspondre à un numéro unique, ce numéro unique doit apparaître en abscisse de l'histogramme H1.

			\item L'histogramme H1 doit avoir une légende qui informe la correspondance des noms de fichier avec leur numéro.
			\item L'histogramme H2 doit avoir autant de couples de barres que d'utilisateurs ayant effectuer un download et upload (de la propriété de l'utilisateur).
			\item L'histogramme H2 doit tracer des couples de barres de couleurs différentes. La couleur représente l'information upload et download.
			\item Le nombre de upload et de download doit apparaître dans la même ordonnée de l'histogramme H2.
			\item Les noms de chaque utilisateur (ayant effectué une opération de "upload" et "download") doit correspondre à un numéro unique, ce numéro unique doit apparaître en abscisse de l'histogramme H2.
			\item L'histogramme H2 doit avoir une légende qui informe la correspondance des noms de fichier avec leur numéro.
			\item L'histogramme H3 doit avoir autant de barre que de tranches horaires (24 divisé par la durée de la tranche horaire).
			\item Les barres de l'histogramme H3 doit être colorier avec une alternance de deux couleurs.
			\item Le nombre de fichier doit apparaître en ordonnée de l'histogramme H3.
			\item Les tranches horaires doivent apparaître en abscisse de l'histogramme H3.

		\end{enumerate}
		
\paragraph{Fichier d'historique des événements}

Un fichier d'historique des événements, appelé aussi fichier "log", permettra au serveur d'enregistrer tous les interactions entre les clients et le serveur lui-même.

	\subparagraph{Fonctionnalité}
		\begin{enumerate}
		
			\item Le serveur doit éditer un fichier d'historique des événements tant que le serveur est en activité.
		
			\item Le serveur doit éditer un nouveau fichier d'historique des événements chaque jour, qui trace les événements passés de 00h00 à 23h59 et 59s.
		
			\item Les fichiers d'historique des événements doivent être stockés dans un répertoire appelé "log", dans le répertoire du logiciel.
		
			\item Les fichiers d'historique des événements doivent être consultables à partir de l'interface graphique du serveur, en cliquant sur un bouton "log" positionné dans la barre d'outil.
			
		\end{enumerate}
		
		\subparagraph{Interface graphique}
			\begin{enumerate}
		
			\item Le bouton "log", suite à un événement "clic" doit ouvrir une fenêtre qui contient un champ de texte multi-ligne et non-éditable, affichant dans ce champ texte le contenu du fichier d'historique des événements de la journée courante.
		
			\item La fenêtre doit avoir pour titre "Historique des événements".
		
			\item La fenêtre doit avoir un calendrier sur lequel l'utilisateur peut cliquer sur une date \textbf{passé} pour consulter le fichier d'historique des événements correspondant.
		
			\item Le calendrier doit être positionné au-dessus du champ de texte.
		
			\item La fenêtre doit avoir un bouton "Fermer" positionné à son angle inférieur droit.
			\item Le champ de texte devrait faire apparaître une barre de défilement verticale et une barre de défilement horizontale si le contenu du fichier d'historique des événements excède les dimensions en hauteur et/ou en largeur du champ de texte.
		\end{enumerate}
		
		\subparagraph{Syntaxe du fichier d'historique des événements}
			\begin{enumerate}
				\item Le fichier d'historique des événements doit être un fichier de texte.
				
				\item Le fichier d'historique des événements doit être nommé de la manière suivante : "log\_file\_DD\_MM\_AAAA". Où DD est le jour, MM le mois et AAAA l'année de l'édition du fichier.
				
				\item Le fichier d'historique des événements doit avoir une extension ".log".
				\item Une ligne du fichier d'historique des événements doit correspondre à un seul événement.
				
				\item La syntaxe d'un événement du fichier d'historique des événements doit etre la suivante : \\
		\textit{"HH:MM:SS nom\_événement précision\_événement client\_ayant\_réalisé\_événement"}\\
		HH:MM:SS :  l'heure du déclenchement de l'événement.\\
		nom\_événement : le nom de l'événement (par exemple "Opération Upload", "Connexion").\\
		précision\_événement : les objets en relation avec les événements (par exemple "Opération Upload \textbf{Document.pdf"}).\\
		client\_ayant\_réalisé\_événement : le nom du client qui a déclenché l'événement.\\		
	\end{enumerate}

\subsubsection{Commun aux deux côtés}
\paragraph{Cryptage}
	\begin{enumerate}
		\item Le programme devrait proposer une fonction de cryptage de fichiers.
		\item Le menu de gestion de fichier devrait proposer un bouton "Cryptage du fichier"
		\item Le programme devrait enregistrer la clé de cryptage de chaque fichier crypté.
		\item Cliquer sur le bouton "Cryptage du fichier" devrait crypter le fichier et enregistrer la clé de cryptage.
		\item Le programme devrait permettre de décrypter les fichiers cryptés.
		\item Le programme devrait utiliser la clé de cryptage correspondante pour décrypter un fichier.
		\item Avant de télécharger un fichier crypté dont la clé de cryptage n'est pas enregistrée, le programme devrait afficher un champ de saisie.
		\item Ce champ de saisie devrait permettre à l'utilisateur d'entrer la clé de cryptage.
		\item Lors de l'envoie de la clé de cryptage, le programme devrait enregistrer celle-ci.
		\item Le programme devrait permettre de communiquer les clés de cryptage entre utilisateurs et administrateur.
		\item Le programme doit crypter les mots de passe des utilisateurs
	\end{enumerate}
	
\paragraph{Communication réseau}

	\begin{enumerate}
		\item Le client et le serveur doivent utiliser le \textbf{protocole FTP} pour échanger les données.
	\end{enumerate}
	
\paragraph{Module statistique }

	\subparagraph{Interface graphique}
	
		\begin{enumerate}

			\item Le module statistique doit être accessible à partir d'un bouton "Statistique" dans la barre d'outil de la fenêtre principale.

			\item Cliquer sur le bouton "Statistique" doit ouvrir une fenêtre (la fenêtre qui contiendra les informations du module statistique).

			\item Cliquer sur le bouton "Statistique" doit initialiser les calculs statistiques (\textcolor{blue}{cf exigences 3.1.1.7}) permettant d'avoir les informations statistiques au moment de l'événement "clic". Cela permet à l'utilisateur d'afficher les informations statistiques en temps réel.

			\item La fenêtre doit avoir un titre "Statistique".

			\item La fenêtre doit avoir une barre d'outil.
	
			\item La barre d'outil doit avoir un bouton "Exporter" positionné à son extrémité gauche.

			\item Cliquer sur le bouton "Exporter" doit permettre la génération d'un rapport de statistique contenant les informations qu'affiche la fenêtre, \textbf{en priorité les graphiques}, dans un format html.

			\item La fenêtre doit avoir un bouton "Fermer" dont l'emplacement sera l'angle inférieur droit de la fenêtre. 

		\end{enumerate}
	

\subsection{Spécification des cas d’utilisation}

\subsubsection{Diagramme de cas d'utilisation}

\begin{center}
	\includegraphics[scale=0.5]{Ressources/Client.png}\\
	\includegraphics[scale=0.5]{Ressources/Serveur.png}
\end{center}


\subsubsection{Utilisabilité}
	\begin{enumerate}
		\item Après une formation d'une demi-heure, L'utilisateur ne doit pas faire plus de 10 erreurs de manipulation sur une durée de 4h.
		\item Après 40 heures de pratique, l'utilisateur ne doit pas faire plus d'une erreur de manipulation sur une durée de 100h.
	\end{enumerate}

\subsubsection{Fiabilité}

\paragraph{Fiabilité du fonctionnement général du logiciel}

	\begin{enumerate}
		\item Le client ne doit pas présenter de bug pendant les 5 premiers mois après le déploiement.
	
		\item Le serveur doit être capable de fonctionner sans défaillance (c'est-à-dire sans que des bugs surgissent) avec 100 clients connectés simultanément pendant une durée de 24h.
		
		\item Le serveur doit être capable de fonctionner plus d'une semaine en condition normale (c'est à dire sans problème de réseau, de surcharge de serveur etc...).	
		
		\item Le serveur doit être capable de fonctionner plus de 48h en condition critique (c'est-à-dire bande passante faible, plus de 200 clients connectés simultanément, plus de 100 téléchargements simultanés, etc ...).
	\end{enumerate}
	
	
\paragraph{Fiabilité des Uploads et des Downloads}
	\begin{enumerate}
		\item Le logiciel ne doit pas échouer plus de 2 transferts de fichiers (opération de "upload" et "download") sur 100, en condition normale (c'est à dire sans problème de réseau, de surcharge de serveur etc...) pour 1 utilisateur.
		
		\item Le logiciel ne doit pas échouer plus de 20 transferts de fichiers (opération de "upload" et "download") sur 100, en condition normale, pour 100 utilisateurs.
		
		\item Le logiciel ne doit jamais se tromper de destination lors d'un transfert de donnée.
	\end{enumerate}		 

\paragraph{Fiabilité des authentifications}

	\begin{enumerate}
	
		\item Le logiciel ne doit pas échouer plus de 1 authentification sur 100 en condition normale.
		
		\item \textbf{Le logiciel ne doit jamais autoriser l'accès des services du serveur d'un client dont le pseudo et/ou mot de passe sont erronés.}
		
		\item Le logiciel ne doit jamais confondre les comptes utilisateurs lors d'une authentification.
		
		\item L'authentification de l'administrateur sur le serveur ne doit jamais échouer.
		
	\end{enumerate}

\subsubsection{Performance}

\paragraph{Simultanéité}

	\begin{enumerate}
		\item Le serveur doit supporter au moins 200 connexions simultanées.
		\item Le serveur doit supporter au moins 20 downloads simultanés et 20 uploads simultanés.
		\item Le client doit supporter au moins 5 downloads simultanés et 5 uploads simultanés.
	\end{enumerate}
	
\paragraph{Temps de réponse}
	
	\begin{enumerate}
		\item Le serveur devrait répondre instantanément à tout type de requêtes avec plus de 100 connexions simultanées.
		\item Le serveur et le client doit réagir instantanément au commande de l'utilisateur. 
	\end{enumerate}

\paragraph{Capacité}

 	\begin{enumerate}
 		\item Le serveur devrait pouvoir gérer au moins 200 comptes utilisateurs.
 	\end{enumerate}


\section{Contraintes de conception}
\begin{description}
\item[Langage:] JAVA.
\item[Outil de developpement:] NetBean.
\end{description}


\section{Sécurité}
Devront se faire de manière sécurisée:
\begin{enumerate}
\item L'enregistrement des mots de passe des utilisateurs dans la base de donnée.
\item Les messages envoyés entre le serveur et chaque client.
\item L'enregistrement des clés de cryptage.
\end{enumerate}


\section{Exigences de documentation utilisateur et d’aide en ligne}
Une documentation utilisateur complète doit être fournie avec le programme.


\section{Classification des exigences fonctionnelles}
\bgroup
\def\arraystretch{1.5}
\begin{tabular}{|c|c|}
	\hline
	{\large \textbf{Fonctionnalité}} & {\large \textbf{Type}}\\
	\hline
	Création de compte & Essentielle\\
	\hline
	Authentification & Essentielle\\
	\hline
	Partage de fichier & Essentielle \\
	\hline
	Utilisation des dossiers & Essentielle\\
	\hline
	Download et Upload & Essentielle \\
	\hline
	Communication en réseau & Essentielle \\
	\hline
	Module de statistique & Souhaitable \\
	\hline
	Fichier d'historique des événements & Souhaitable \\
	\hline
	Limitations & Souhaitable\\
	\hline
	Suppression de compte & Souhaitable \\
	\hline
	Bannissement de compte & Souhaitable \\
	\hline
	Gestionnaire de marque-pages & Souhaitable\\
	\hline
	Cryptage & Optionnelle\\
	\hline
	Compte temporaire & Optionnelle \\
	\hline
	Bannissement temporaires & Optionnelle \\
	\hline
	Bannissement partiel & Optionnelle \\
	\hline
\end{tabular}
\egroup

\section{Annexe}
\subsection{Architecture du système KouldNotShare}
\begin{center}
\includegraphics[scale=0.8]{Ressources/doc_exigence.png}
\end{center}

\end{document}