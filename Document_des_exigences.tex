\documentclass[10pt,a4paper]{report}

\usepackage[utf8]{inputenc}
\usepackage{amsmath}
\usepackage{amsfonts}
\usepackage{amssymb}
\usepackage{enumitem}

\author{Kevin BASCOL, Kevin LAOUSSING, Nicolas REYNAUD}
\title{Document des exigences}

\makeatletter
\renewcommand{\thesection}{\@arabic\c@section}
\makeatother

\begin{document}

\maketitle
\newpage
\setcounter{secnumdepth}{4}
\setcounter{tocdepth}{4}
\renewcommand{\contentsname}{Sommaire}
\tableofcontents
\newpage

\chapter*{Spécification d'exigences logicielles}

\section{Introduction}
[L’introduction donne une vue d’ensemble de tout le document. On y présente toute information que le lecteur a besoin pour  comprendre le document. Elle comprend l’objectif du document, sa portée, les définitions, acronymes et abréviations, les références et une vue d’ensemble du document.
Note : La SEL comporte l’ensemble des exigences logicielles pour une portion ou pour tout le système. La présente spécification est adaptée pour  un projet utilisant une modélisation de cas d’utilisation. Cet artéfact est un paquetage qui comprend les cas d’utilisation du modèle des cas d’utilisation et les spécifications supplémentaires applicables ainsi que les autres informations pertinentes.
Plusieurs aménagements d’une SEL sont possibles. La norme [IEEE830-1998] est la référence pour de plus amples explications ainsi que pour d’autres options d’organisation du document.]

\subsection{Objectif du document}
%[Préciser les objectifs de ce document. La SEL doit décrire le comportement externe de l’application ou du sous-système identifié. Elle décrit aussi les exigences non-fonctionnelles et les autres facteurs nécessaires à une description complète et compréhensible des exigences pour le logiciel.]%
Le but de ce document est de décrire les spécifications requises pour la création d'un  client/serveur FTP pour les équipes de développement de logiciels.
Le public visé par ce document comprend les futurs développeurs du projet et le personnel d'évaluation technique de l'organisation cliente.



\subsection{Portée du document}
%[Une brève description de la portée de ce document, l’application qu’il décrit, les caractéristiques ou autres sous-systèmes auxquels l’application est associée, le ou les modèles de cas d’utilisation qu’il décrit ainsi que tout autre chose qui peut être influencée ou affectée par ce document.]%
Le logiciel de client/serveur FTP est un outil permettant à des particuliers ou des entreprises l'échange des données de manière contrôler. Par exemple, chaque utilisateurs propriétaires d'un fichier stocké dans le serveur FTP du logiciel, pourront paramétrer les droits de téléchargements sur ce fichier, et ainsi ils disposeront du pouvoir de restreindre ou d'augmenter l'accessibilité de son fichier...

\subsection{Définitions, acronymes et abréviations}
[Énumérer les définitions de tous les termes, acronymes et abréviations nécessaires à la compréhension du document d’architecture logicielle. Cette information peut renvoyer à l’artéfact Glossaire du projet..]

\subsection{Références}
 [Cette section comporte la liste de tous les documents cités dans le document. Chaque document doit être identifié par son titre, son numéro, lorsque applicable, sa date et l’organisation qui l’a publiée. Les sources qui peuvent fournir les références doivent être citées. Cette dernière information peut être elle-même une référence à une annexe ou à un autre document.]
 
\subsection{Vue d’ensemble}
[Cette section décrit le contenu du reste du document  et explique comment le document est organisé.]


\section{Description générale}
[Décrire les principaux facteurs qui affectent le produit et ses exigences. On n’y énonce pas des exigences spécifiques, mais on y fournit une toile de fond aux exigences qui sont définies en détail à la section 3 afin d’en faciliter la compréhension. Cela comprend les items suivants :]

\subsection{Perspectives du produit}

\subsubsection{Interfaces système}
[Décrire les interfaces qui permettent la communication avec d’autres systèmes.]

\subsubsection{Interfaces utilisateurs}
[Décrire les interfaces utilisateur ou référer au prototype d’interface utilisateur.]

\subsubsection{Interfaces matérielles}
[Définir les interfaces entre le logiciel et le matériel, incluant la structure logique, les adresses physiques, le comportement attendu, etc.]

\subsubsection{Interfaces logicielles}
[Décrire les interfaces logicielles avec les autres composants du système. Ce peut être des composants achetés. Des composants de d’autres applications qui sont réutilisés ou des composants développés pour des sous-systèmes hors de la protée de la présente SEL et qui interagissent avec le présent système..]

\subsubsection{Interfaces de communication}
[Décrire les interfaces de communication avec les autres systèmes comme les réseaux locaux, les serveurs distants, etc.]

\subsubsection{Contraintes de mémoire}
[Indiquer les besoins en mémoire primaire et secondaire.]

\subsection{Fonctions du produit}
[Décrire brièvement les fonctions principales du logiciel. Par exemple :
Les fonctions d’un système de gestion peuvent être la maintenance d’un compte client, accéder à l’état de compte du client et produire la facturation.]

\subsection{Caractéristiques des utilisateurs}
[Décrire les caractéristiques générales des utilisateurs qui ont un impact sur les exigences du document. Cela inclut le niveau de scolarité, l’expérience et l’expertise technique.]

\subsection{Contraintes}
[Décrire toute autre contrainte qui peut limiter le développement du système et qui n’apparaissent pas dans les autres sections de la SEL.]

\subsection{Hypothèses et dépendances}
[Décrire tout élément de faisabilité technique, disponibilité de sous-système ou de composant ou toute autre hypothèse liée au projet de laquelle dépend la viabilité du logiciel.]

\subsection{Exigences reportées}
[Énumérer les exigences qui peuvent être réalisées dans des versions futures du système.]

\section{Exigences spécifiques}
[Énumérer et décrire les exigences logicielles avec des détails suffisants à la compréhension pour permettre aux concepteurs de concevoir le système et aux testeurs de s’assurer que le système satisfait ces exigences. Lorsque la modélisation par cas d’utilisation est utilisée, ces exigences peuvent être incluses dans les cas d’utilisation et dans les spécifications supplémentaires applicables.]

\subsection{Fonctionnalités}
[Décrire les exigences fonctionnelles du système qui peuvent être exprimées et langage naturel. Pour plusieurs applications, c’est la partie principale de la SEL et son organisation doit, par conséquent, être bien réfléchie. Elle est habituellement hiérarchisée par caractéristiques, mais elle peut l’être, par utilisateur ou par sous-système. Les exigences fonctionnelles peuvent inclure les caractéristiques, les capacités et la sécurité.
Lorsque des outils de développement, tels des référentiels d’exigences ou des outils de modélisation sont utilisés, on peut référer à ces données en indiquant l’endroit et le nom de cet outil]
\subsubsection{\textless Communication en réseau\textgreater}

\paragraph{\textbf{Procédure d'initialisation de la communication Client-Serveur}}

\begin{itemize}[label = $\triangleright$]
\item  Du démarrage jusqu'à sa fermeture, le serveur doit être à l'écoute sur le port \textbf{port 21}.

\item Le client doit se connecter sur le \textbf{port 21} du serveur pour pouvoir échanger des données.

\item Le client doit utiliser un port libre c'est-à-dire que son numéro de port doit être supérieur à 1024.

\item Le client et le serveur doivent utiliser le protocole FTP pour échanger les données.

\item Un témoin graphique (de type diode) doit signaler la réussite de la connexion si la connexion est établie. 
\end{itemize}

\paragraph{\textbf{Procédure en cas d'impossibilité d'initiation de la communication Client-Serveur}}

\begin{itemize}[label = $\triangleright$]
\item Le client doit prévenir l'utilisateur s'il ne parvient pas à établir une communication avec le serveur à l'aide d'une fenêtre de dialogue affichant les messages suivant : 
\indent

\begin{itemize}
\item \textit{"Impossible d'établir une communication avec le serveur $\textless nom\_du\_serveur\textgreater$:\\
-Vérifiez l'état de votre connexion Internet.\\
-Vérifiez que le nom du serveur n'est pas erroné."\\}
Si la source du problème provient du réseau (problème de routage, connexion coupée) ou si le serveur est inexistant.

\item \textit{"Impossible d'établir une communication avec le serveur $\textless nom\_du\_serveur\textgreater$: le serveur est surchargé."\\}
Si le nombre de connexion autorisé par le serveur est atteint.

\item \textit{"Impossible d'établir une communication avec le serveur $\textless nom\_du\_serveur\textgreater$:\\
$\textless nom\_du\_serveur\textgreater$ vous a banni !"\\}
Si le client a été auparavant banni par le serveur. 
\end{itemize}

\item La fenêtre de dialogue doit avoir un bouton "Fermer" pour fermer celle-ci.

\item La fenêtre de dialogue doit avoir un bouton "Ré-essayer" pour que le client puisse retenter une connexion avec le serveur avec les mêmes paramètres de communications que la tentative précédente.
\end{itemize}


\paragraph{\textbf{Procédure en cas de coupure de la communication Client-Serveur}}

\begin{itemize}[label = $\triangleright$]
\item Le client doit prévenir l'utilisateur si la communication avec le serveur est interrompue à l'aide d'une fenêtre de dialogue affichant le message suivant:\\
\textit{"Connexion interrompue avec le serveur $\textless nom\_du\_serveur\textgreater$."}

\item La fenêtre de dialogue doit avoir un bouton "Fermer".

\item Le client et le serveur doivent automatiquement fermer le port qui était dédié à la communication interrompue.
\end{itemize}

\paragraph{\textbf{Paramètre de la communication Client-Serveur}}

\begin{itemize}[label = $\triangleright$]
\item Le serveur doit respecter la limite de débit de téléchargement configurée préalablement par son administrateur.
\item Le client doit pouvoir télécharger les données avec le débit imposé par le serveur par défaut.
\item L'utilisateur du client devrait pouvoir restreindre le débit de téléchargement à partir du débit maximal de téléchargement autorisé par le serveur. 
\end{itemize}

\paragraph{\textbf{Procédure de fermeture de la communication Client-Serveur}}

\begin{itemize}[label = $\triangleright$]
\item Le client doit \textbf{clôturer} la communication en cas d'\textbf{upload}.

\item Le serveur doit \textbf{clôturer} la communication en cas de \textbf{download}. 
\end{itemize}

\subsubsection{Module de statistique}

Le serveur et le client doivent contenir un module de statistique.


\paragraph{\textbf{Serveur\\}}

Ce module permettra à l'administrateur de contrôler les activités sur son serveur, et de pouvoir par la suite de modifier la configuration de son serveur pour mieux adapter celui-ci au besoin des utilisateur.

Voici les exigences :

\begin{itemize}[label = $\triangleright$]
\item Le module de statistique doit mesurer la fréquence de connexion des utilisateurs clients.

\item Le module de statistique doit comptabiliser le nombre d'opération de "download" de chaque fichier stocké dans le serveur.

\item Le module de statistique doit comptabiliser le nombre d'opération de "download" par utilisateur.

\item Le module de statistique doit comptabiliser le nombre d'opération de "upload" par utilisateur.

\item Le module de statistique doit comptabiliser le nombre de téléchargement par tranche horaire (2 heures).

\item Le module de statistique doit pouvoir afficher en temps réel en plus d'éditer un rapport toute les 2 semaines contenant les statistiques suivante:

\indent
\begin{itemize}
\item le fichier le plus télécharger.

\item l'utilisateur qui "upload" le plus de fichiers.

\item l'utilisateur qui "download" le plus de fichiers.

\item un histogramme qui trace la fréquence de téléchargement de tous les fichiers stockés dans le serveur.

\item un histogramme qui trace le nombre d'opération de "download" et de "upload" de chaque utilisateur.

\item un histogramme qui trace la moyenne du nombre d'opération de "download" et de "upload" par tranche de 2 heures.
\end{itemize}

\item Le rapport en temps réel doit d'être accessible via un bouton sur l'interface graphique du programme.

\item Le rapport édité par le module de statistique doit être un fichier de format html (le navigateur pourra interpréter le fichier et tracer les graphiques).
\end{itemize}

\paragraph{\textbf{Client\\}}

Ce module permettra à l'utilisateur de contrôler son activité sur le serveur FTP, et de contrôler les opérations faites sur ces fichiers par les autres utilisateurs.

Voici les exigences :
\begin{itemize}[label = $\triangleright$]
\item Le module de statistique doit comptabiliser le nombre d'opération "download" de chacun de ses fichiers (les fichiers dont il est le propriétaire), par utilisateur (ceux qui téléchargent ses fichiers).
\item Le module de statistique doit afficher un rapport en temps réel contenant: 

\begin{itemize}
\item le fichier le plus télécharger.

\item l'utilisateur qui "download" le plus ses fichiers.

\item un histogramme qui trace la fréquence de téléchargement de tous ses fichiers en fonction des utilisateurs qui les ont téléchargés.
\end{itemize} 
\end{itemize}


\subsection{Spécification des cas d’utilisation}
[Lorsqu’il y a une modélisation par cas d’utilisation, ceux-ci décrivent la majorité des exigences fonctionnelles du système ainsi que certaines exigences non-fonctionnelles. On peut référer au document de Spécification des cas d’utilisation.]

\subsection{Exigences supplémentaires}
[Décrire les exigences qui ne sont pas incluses dans les cas d’utilisation ainsi que les exigences non-fonctionnelles. On peut référer au document de Spécifications supplémentaires.]

\subsubsection{Utilisabilité}
[Décrire les exigences qui affectent l’utilisabilité comme, par exemple:\\
•	Le temps de formation nécessaire à un utilisateur normal ou expert avant d’être productif.\\
•	Les temps d’exécution pour les tâches courantes\\
•	Les exigences pour satisfaire aux standards d’utilisabilité d’interface graphique de, par exemple, Microsoft.]
\paragraph{\textless Nom de l’exigence d’utilisabilité 1 \textgreater}
[Description de l’exigence]

\subsubsection{Fiabilité}
[Décrire les exigences qui affectent la fiabilité comme, par exemple:\\
•	La disponibilité: le pourcentage d’heures d’utilisation, les périodes de maintenance, mode d’opération lors de dégradation, etc.\\
•	Durée moyenne de fonctionnement avant défaillance, exprimée en heures, en jours, en mois ou en années.\\
•	Durée moyenne de rétablissement, qui est le délai moyen de réparation d'une unité fonctionnelle après une défaillance.\\
•	Exactitude. précision, souvent définie par de normes, requise pour les extrants.\\
•	Nombre maximum d’anomalies exprimé habituellement en KLOC, en défaut par millier de ligne de code ou par points de fonction.\\
•	Criticité d’anomalie, mineure, significative, critique en décrivant ce que critique signifie.]
\paragraph{\textless Nom de l’exigence de fiabilité 1 \textgreater}
[Description de l’exigence]

\subsubsection{Performance}
[Décrire les caractéristiques de la performance du système. Référer les cas d’utilisation lorsque applicable.\\
•	Temps de réponse par transaction (moyen, maximum)\\
•	Débit (transactions par seconde)\\
•	Capacité (nombre de client ou de transaction que le système doit supporter)\\
•	Mode d’opération lors de dégradation (Mode d’opération acceptable lorsque la performance du système se détériore)\\
•	Utilisation de ressources (mémoire, disque, communications, etc.]
\paragraph{\textless Nom de l’exigence de performance 1 \textgreater}
[Description de l’exigence]

\subsubsection{Maintenabilité}
[Décrire les exigences qui permettent d’assurer le support et la maintenabilité du système comme, par exemple, les normes de codage, le conventions d’identification, le sbibliothèque3s de classe, l’accès à la maintenance, les services de maintenances, etc. 
\paragraph{\textless Nom de l’exigence de maintenance 1 \textgreater}
[Description de l’exigence]


\section{Contraintes de conception}
[Indiquer les contraintes de conception comme le langage de programmation, le processus de développement logiciel, les outils de développement, les contraintes d’architecture, les composants achetés, les bibliothèques de classe, etc..]

\subsection{\textless Nom de la contrainte de conception 1 \textgreater}
[Description de l’exigence]


\section{Sécurité}
[Identifier les données qui doivent être protégées et le type de menace auquel elles sont exposées comme, par exemple, menaces physiques, types de personnes qui peuvent être la source de menace, les exigences d’accès au système, l’encryptage des données, la vérifiabilité qui est la détection des anomalies et des opérations frauduleuses. 
Énumérer la liste des objets qui doivent faire l’objet d’une protection physique ou logique]


\section{Exigences de documentation utilisateur et d’aide en ligne}
[Décrire les exigences de documentation utilisateur et  d’aide du système.]


\section{Normes applicables}
[Décrire par référence toutes normes applicables et les sections précises de ces normes qui s’appliquent au système. Cela inclut, par exemple, les normes de qualité, légales ou réglementaire, les normes industrielles d’utilisabilité, l’interopérabilité, les normes d’internationalisation, conformité au système d’exploitation, etc. ]


\section{Classification des exigences fonctionnelles}
[Énumérer dans un tableau toutes les exigences fonctionnelles et leur type, essentielle, souhaitable ou optionnelle. Elles peuvent être triées par leur ordre d’apparition dans le document ou par type.
Fonctionnalité	Type
...	
...	


\section{Annexes}
[Lorsqu’on utilise des annexes, il faut explicitement indiquer si elles font partie des exigences.]

\end{document}