\documentclass[10pt,a4paper,landscape]{report}

\usepackage[utf8]{inputenc}
\usepackage{amsmath}
\usepackage{amsfonts}
\usepackage{amssymb}
\usepackage{graphicx}
\usepackage{xcolor}
\definecolor{gris}{rgb}{0.75,0.75,0.75}
\usepackage{colortbl}
\usepackage{enumitem}
\usepackage{version}
\usepackage{multirow}
\usepackage[top=1cm, bottom=2cm, left=2cm, right=2cm]{geometry}
\usepackage{fancyhdr}
\pagestyle{fancy}

\fancyhead{}
\fancyfoot{} 
\lhead{\includegraphics{../Logo/logoKNKMini.jpg} \hspace{0.1cm} Kould Not Konect  \hspace{0.4cm} \vline}
\chead{Cahier des tests de recette}
\rhead{Kould Not Share}
\rfoot{\thepage}

\author{Nicolas REYNAUD, Kevin BASCOL}
\title{Cahier des tests de recette}
\date{5 Novembre 2014}

\makeatletter
\renewcommand{\thesection}{\@arabic\c@section}
\makeatother

\makeatletter
\def\chline#1{%
  \noalign{\ifnum0=`}\fi\begingroup\color{#1}\hrule \@height \arrayrulewidth\endgroup \futurelet
   \reserved@a\@xhline}
\def\ccline#1#2{\@ccline#1\@nil#2\@nil}
\def\@ccline#1-#2\@nil#3\@nil{%
  \omit
  \@multicnt#1%
  \advance\@multispan\m@ne
  \ifnum\@multicnt=\@ne\@firstofone{&\omit}\fi
  \@multicnt#2%
  \advance\@multicnt-#1%
  \advance\@multispan\@ne
  \begingroup\color{#3}\leaders\hrule\@height\arrayrulewidth\hfill\endgroup
  \cr
  \noalign{\vskip-\arrayrulewidth}}
\makeatother

\begin{document}
\makeatletter
	\begin{titlepage}
	
	\begin{figure}
		\begin{minipage}[c]{.46\linewidth}
		\end{minipage} \hfill
		\begin{minipage}[c]{.20\linewidth}
			\begin{center}
				\includegraphics{../Logo/logoKNK.jpg}\\
				{\large Kould Not Konect}
			\end{center}
		\end{minipage}
	\vspace{1cm}
	\end{figure}
	
	\centering
		{
		\hrule height 2pt
		\vspace{0.7cm}
		\Huge \textbf{\@title}}\\
		\vspace{0.7cm}
		\hrule height 2pt
		\vspace{1.5cm}
		{\LARGE  Projet \textbf{Kould Not Share} v1.0}
		
		\vfill
		
		\begin{tabular}{|c|c|c|}
			\hline
			Version & Date & Description\\
			\hline
			v.1 & 05/11/14 & Tests de la première version des exigences\\
			\hline
			 & & \\
			\hline
			 & & \\
			\hline
		\end{tabular}\\
		\vspace{1cm}
		\@author\\
		\end{titlepage}
\makeatother
\setcounter{secnumdepth}{5}
\setcounter{tocdepth}{5}
\renewcommand{\contentsname}{Sommaire}
\begingroup\makeatletter
\def\@makeschapterhead#1{%
  {\parindent \z@ \raggedright
    \normalfont
    \interlinepenalty\@M
    \Huge \bfseries  #1\par\nobreak
    \vskip 20pt% <---- à réduire pour avoir plus de place
  }}\makeatother
\tableofcontents
\endgroup
\thispagestyle{empty}
\setcounter{page}{0}
\newpage

\newgeometry{top=2cm, bottom=2cm, left=0.5cm, right=0.5cm}

\section{Tests Client}
\subsection{Authentification}
\begin{center}
	\bgroup
	\def\arraystretch{1.5}
	\begin{tabular}{|p{2.5cm}|p{2cm}|p{8cm}|p{8cm}|p{5cm}|}
		\hline
		\rowcolor{gris}Référence & Exigence & Test & Résultat attendu & Résultat/Commentaires\\
		\hline
		AuthC.1 & 3.1.1.1.1.1 3.1.1.1.1.2 3.1.1.1.1.3 3.1.1.1.1.13 & Ouvrir le logiciel. & Le formulaire de connexion apparait en pop-up au centre de la fenêtre, le reste de l'application est grisé (hormis la barre d'action). & \\
		\cline{1-2}\ccline{3-3}{gris}\cline{4-5}
		AuthC.1.1 & 3.1.1.1.1.4 3.1.1.1.1.5 & Cliquer sur le bouton de déconnexion. & Le bouton doit être grisé et ne doit rien déclencher. & \\
		\cline{1-2}\ccline{3-3}{gris}\cline{4-5}
		AuthC.1.2 & 3.1.1.1.1.6 3.1.1.1.1.7 3.1.1.1.1.8 &  & sur le pop-up, le titre "Authentification" apparait centré et est au dessus d'une barre horizontale. & \\
		\cline{1-2}\cline{4-5}
		AuthC.1.3 & 3.1.1.1.1.9 3.1.1.1.1.10 & & Les mots "Serveur", "Nom d'utilisateur" et "Mot de passe" sont affichés, chacun suivit d'un champ de saisie de texte. & \\
		\cline{1-2}\cline{4-5}
		AuthC.1.4 & 3.1.1.1.1.11 & & Un bouton "Se connecter" est affiché centré en bas du pop-up & \\
		\cline{1-2}\ccline{3-3}{gris}\cline{4-5}
		AuthC.1.4.1 & 3.1.1.1.1.11 & Remplir incorrectement un ou plusieurs champs de saisie puis cliquer sur le bouton "Se connecter" (à effectuer avec plusieurs combinaison de champs incorrects). & Le(s) champ(s) incorrect(s) possède(nt) une bordure rouge. & \\
		\cline{1-2}\ccline{3-3}{gris}\cline{4-5}
		AuthC.1.4.2 & 3.1.1.1.2.1 & Remplir correctement les champs, puis cliquer sur "Se connecter". & Le programme se connecte au serveur avec l'identifiant donné. & \\
		\hline
	\end{tabular}
	\egroup
\end{center}

\subsection{Gestionnaire de marques-page}
\begin{center}
	\bgroup
	\def\arraystretch{1.5}
	\begin{tabular}{|p{2.5cm}|p{2cm}|p{8cm}|p{8cm}|p{5.5cm}|}
		\hline
		\rowcolor{gris}Référence & Exigence & Test & Résultat attendu & Résultat/Commentaires\\
		\hline
		GdmpC.1 & 3.1.1.2.1 3.1.1.2.2 & Ouvrir le logiciel. & Un menu déroulant "Bookmarks" est présent dans la barre d'action. & \\
		\cline{1-2}\ccline{3-3}{gris}\cline{4-5}
		GdmpC.1.1 & 3.1.1.2.3 3.1.1.2.7& Dérouler le menu "Bookmarks". & Les boutons "Afficher les marque-pages" et "Mettre un marque-page" sont présents dans le menu. & \\
		\cline{1-2}\ccline{3-3}{gris}\cline{4-5}
		GdmpC.1.1.1 & 3.1.1.2.4 & Cliquer sur "Afficher les marque-pages". & un pop-up s'ouvre avec la liste des liens vers les serveurs affichés avec les noms des marque-pages & \\
		\cline{1-2}\ccline{3-3}{gris}\cline{4-5}
		GdmpC.1.1.2 & 3.1.1.2.8 & Cliquer sur "Mettre un marque-page". & Un champs de saisie "nom" et un case à cocher "enregistrer mot de passe" apparaissent dans une boite de dialogue. & \\
		\cline{1-2}\ccline{3-3}{gris}\cline{4-5}
		GdmpC.1.1.2.1 & 3.1.1.2.9 & Entrer un nom. & Le marque page est ajouté à la liste. & \\
		\cline{1-2}\ccline{3-3}{gris}\cline{4-5}
		GdmpC.1.1.2.2 & 3.1.1.2.10 & Entrer un nom et cocher la case. & Le marque-page est ajouté à la liste et permet une connexion directe. & \\
		\hline
		GdmpC.2 & 3.1.1.2.5 & Ouvrir le fichier des marque-pages. & Les marque-pages sont composés d'un nom, de l'adresse du serveur et de l'identifiant de l'utilisateur. & \\
		\cline{1-2}\cline{4-5}
		GdmpC.2.1 & 3.1.1.2.6 & & Les marque-pages où le mot de passe est enregistré sont composé aussi du mot de passe crypté. & \\
		\hline
	\end{tabular}
	\egroup
\end{center}

\subsection{Limitations}
\begin{center}
	\bgroup
	\def\arraystretch{1.5}
	\begin{tabular}{|p{2.5cm}|p{2cm}|p{8cm}|p{8cm}|p{5cm}|}
		\hline
		\rowcolor{gris}Référence & Exigence & Test & Résultat attendu & Résultat/Commentaires\\
		\hline
		LimiC.1 & 3.1.1.3.1 & Ouvrir le logiciel. & Le menu déroulant "Configuration est présent dans la barre d'action. & \\
		\cline{1-2}\ccline{3-3}{gris}\cline{4-5}
		LimiC.1.1 & 3.1.1.3.3 3.1.1.3.7 3.1.1.3.11 3.1.1.3.14 & Cliquer sur le menu déroulant "Configuration". & Les boutons "Limiter vitesse téléchargement", "Limiter vitesse téléversement", "Plages horaires téléchargement" et "Plages horaires téléversement" sont présent dans le menu déroulant. & \\
		\cline{1-2}\ccline{3-3}{gris}\cline{4-5}
		LimiC.1.1.1 & 3.1.1.3.4 3.1.1.3.5 & Cliquer sur "Limiter vitesse téléchargement". & Une boite de dialogue avec un curseur apparait. & \\
		\cline{1-2}\ccline{3-3}{gris}\cline{4-5}
		LimiC.1.1.1.2 & 3.1.1.3.2 & Choisir une limite et valider la boite de dialogue. & La vitesse de téléchargement ne dépasse pas la limite fixée. & \\
		\cline{1-2}\ccline{3-3}{gris}\cline{4-5}
		LimiC.1.1.2 & 3.1.1.3.8 3.1.1.3.9 & Cliquer sur "Limiter vitesse téléversement". & Une boite de dialogue avec un curseur apparait. & \\
		\cline{1-2}\ccline{3-3}{gris}\cline{4-5}
		LimiC.1.1.2.1 & 3.1.1.3.6 & Choisir une limite et valider la boite de dialogue. & La vitesse de téléversement ne dépasse pas la limite fixée. & \\
		\cline{1-2}\ccline{3-3}{gris}\cline{4-5}
		LimiC.1.1.3 & 3.1.1.3.12 & Cliquer sur "Plages horaires téléchargement". & Une boite de dialogue apparait avec la liste des heures du jours couplées à une case à cocher. & \\
		\cline{1-2}\ccline{3-3}{gris}\cline{4-5}
		LimiC.1.1.3.1 & 3.1.1.3.10 & Sélectionner des horaires et valider la boite de dialogue. & Aux heures sélectionnées le programme bloque le téléchargement. & \\
		\cline{1-2}\ccline{3-3}{gris}\cline{4-5}
		LimiC.1.1.4 & 3.1.1.3.15 & Cliquer sur "Plages horaires téléversement". & Une boite de dialogue apparait avec la liste des heures du jours couplées à une case à cocher. & \\
		\cline{1-2}\ccline{3-3}{gris}\cline{4-5}
		LimiC.1.1.4.1 & 3.1.1.3.13 & Sélectionner des horaires et valider la boite de dialogue. & Aux heures sélectionnées le programme bloque le téléversement. & \\
		\hline
	\end{tabular}
	\egroup
\end{center}

\subsection{Utilisation des dossiers}
\begin{center}
	\bgroup
	\def\arraystretch{1.5}
	\begin{tabular}{|p{2.5cm}|p{2cm}|p{8cm}|p{8cm}|p{5cm}|}
		\hline
		\rowcolor{gris}Référence & Exigence & Test & Résultat attendu & Résultat/Commentaires\\
		\hline
		UtdoC.1 & 3.1.1.4.1 & Ouvrir le logiciel. & L'arborescence des dossiers dont l'utilisateur a accès sur le serveur est affichée sur la droite de la fenêtre. & \\
		\cline{1-2}\ccline{3-3}{gris}\cline{4-5}
		UtdoC.1.1 & 3.1.1.4.2 3.1.1.4.3 & Cliquer sur les dossiers dans l'arborescence. & Les sous-dossiers et fichiers des dossiers sur lesquels on a cliqué sont affichés dans l'arborescence. & \\
		\cline{1-2}\ccline{3-3}{gris}\cline{4-5}
		UtdoC.1.2 & 3.1.1.4.5 3.1.1.4.6 3.1.1.4.10 3.1.1.4.13 & Faire un clic droit sur un dossier. & Un menu de gestion de dossier apparait, les boutons "Nouveau sous-dossier", "Supprimer le dossier", et "Nouveau fichier vide" sont disponible à l'intérieur. & \\
		\cline{1-2}\ccline{3-3}{gris}\cline{4-5}
		UtdoC.1.2.1 & 3.1.1.4.7 3.1.1.4.8 & Cliquer sur "Nouveau sous-dossier". & Une boite de dialogue apparait avec un champ de saisie "nom". & \\
		\cline{1-2}\ccline{3-3}{gris}\cline{4-5}
		UtdoC.1.2.1.1 & 3.1.1.4.4 & Entrer un nom et valider la boite de dialogue. & Un nouveau sous-dossier avec le nom donné dans la boite de dialogue apparait dans l'arborescence. & \\
		\cline{1-2}\ccline{3-3}{gris}\cline{4-5}
		UtdoC.1.2.2 & 3.1.1.4.11 3.1.1.4.11.1 & Cliquer sur "Supprimer le dossier". & Une boite de dialogue proposant un bouton "Valider" et un "Annuler" apparait. & \\
		\cline{1-2}\ccline{3-3}{gris}\cline{4-5}
		UtdoC.1.2.2.1 & 3.1.1.4.9 3.1.1.4.11.2 & Cliquer sur "Valider". & La boite de dialogue disparait et le dossier est supprimé. & \\
		\cline{1-2}\ccline{3-3}{gris}\cline{4-5}
		UtdoC.1.2.2.2 & 3.1.1.4.11.3 & Cliquer sur "Annuler". & La boite de dialogue disparait sans rien déclencher. & \\
		\cline{1-2}\ccline{3-3}{gris}\cline{4-5}
		UtdoC.1.2.3 & 3.1.1.4.12 3.1.1.4.14 & Cliquer sur "Nouveau fichier vide". & Un nouveau fichier "nouveau\_fichier" apparait dans le dossier. & \\
		\cline{1-2}\ccline{3-3}{gris}\cline{4-5}
		Utdoc.1.3 & 3.1.1.4.16 3.1.1.4.17 & Faire un clic droit sur un des fichiers dans l'arborescence. & Un menu de gestion de fichier apparait avec un bouton "Supprimer le fichier" à l'intérieur. & \\
		\cline{1-2}\ccline{3-3}{gris}\cline{4-5}
		UtdoC.1.2.2 & 3.1.1.4.18 3.1.1.4.18.1 & Cliquer sur "Supprimer le fichier". & Une boite de dialogue proposant un bouton "Valider" et un "Annuler" apparait. & \\
		\cline{1-2}\ccline{3-3}{gris}\cline{4-5}
		UtdoC.1.2.2.1 & 3.1.1.4.15 3.1.1.4.18.2 & Cliquer sur "Valider". & La boite de dialogue disparait et le dossier est supprimé. & \\
		\cline{1-2}\ccline{3-3}{gris}\cline{4-5}
		UtdoC.1.2.2.2 & 3.1.1.4.18.3 & Cliquer sur "Annuler". & La boite de dialogue disparait sans rien déclencher. & \\
		\hline
	\end{tabular}
	\egroup
\end{center}

\subsection{Communication réseau}

\subsubsection{Procédure d'initialisation de la communication}

\begin{center}
	\bgroup
	\def\arraystretch{1.5}
	\begin{tabular}{|p{2.5cm}|p{2cm}|p{8cm}|p{8cm}|p{5.5cm}|}
		\hline
		\rowcolor{gris}Référence & Exigence & Test & Résultat attendu & Résultat/Commentaires\\
		\hline
		\hline
	\end{tabular}
	\egroup
\end{center}

\subsubsection{Procédure en cas d'échec d'initialisation de la communication}

\begin{center}
	\bgroup
	\def\arraystretch{1.5}
	\begin{tabular}{|p{2.5cm}|p{2cm}|p{8cm}|p{8cm}|p{5.5cm}|}
		\hline
		\rowcolor{gris}Référence & Exigence & Test & Résultat attendu & Résultat/Commentaires\\
		\hline
		\hline
	\end{tabular}
	\egroup
\end{center}

\subsubsection{Procédure en cas de coupure de la communication}

\begin{center}
	\bgroup
	\def\arraystretch{1.5}
	\begin{tabular}{|p{2.5cm}|p{2cm}|p{8cm}|p{8cm}|p{5.5cm}|}
		\hline
		\rowcolor{gris}Référence & Exigence & Test & Résultat attendu & Résultat/Commentaires\\
		\hline
		\hline
	\end{tabular}
	\egroup
\end{center}

\subsubsection{Procédure en cas de fermeture de la communication}

\begin{center}
	\bgroup
	\def\arraystretch{1.5}
	\begin{tabular}{|p{2.5cm}|p{2cm}|p{8cm}|p{8cm}|p{5.5cm}|}
		\hline
		\rowcolor{gris}Référence & Exigence & Test & Résultat attendu & Résultat/Commentaires\\
		\hline
		\hline
	\end{tabular}
	\egroup
\end{center}

\subsection{Download et Upload}

\begin{center}
	\bgroup
	\def\arraystretch{1.5}
	\begin{tabular}{|p{2.5cm}|p{2cm}|p{8cm}|p{8cm}|p{5.5cm}|}
		\hline
		\rowcolor{gris}Référence & Exigence & Test & Résultat attendu & Résultat/Commentaires\\
		\hline
		DwupC.1 & 3.1.1.6.1 & Connecter le client à un serveur FTP autre qu'un serveur KNS. & La connexion doit s'effectuer. & \\
		\cline{1-2}\ccline{3-3}{gris}\cline{4-5}
		DwupC.1.1 & 3.1.1.6.3 & Uploader un fichier. & Le fichier est disponible sur le serveur. & \\
		\hline
		DwupC.2 & 3.1.1.6.2 & Connecter le client à un serveur FTP KNS avec un compte invalide & La connexion ne doit pas s'effectuer. & \\
		\hline
		DwupC.3 & 3.1.1.6.2 & Connecter le client à un serveur FTP KNS avec un compte valide & La connexion doit s'effectuer. & \\
		\cline{1-2}\ccline{3-3}{gris}\cline{4-5}
		DwupC.3.1 & 3.1.1.6.4 3.1.1.6.6 & Uploader un fichier, dont l'extension n'est pas interdite. & Une barre de progression s'affiche, arrivée au bout le fichier est disponible sur le serveur. & \\
		\cline{1-2}\ccline{3-3}{gris}\cline{4-5}
		DwupC.3.2 & 3.1.1.6.5 & Uploader un fichier, dont l'extension est interdite. & Un boite de dialogue apparait avec le mention "Le transfert du \textless fichier\_cible\_a\_télécharger \textgreater n'est pas autorisé par le serveur \textless nom\_du\_serveur \textgreater". & \\
		\hline
		\rowcolor{red}DwupC.4 & 3.1.1.6.7 & & & \\		
		\hline	
	\end{tabular}
	\egroup
\end{center}

\subsection{Module de statistique}

\begin{center}
	\bgroup
	\def\arraystretch{1.5}
	\begin{tabular}{|p{2.5cm}|p{2cm}|p{8cm}|p{8cm}|p{5.5cm}|}
		\hline
		\rowcolor{gris}Référence & Exigence & Test & Résultat attendu & Résultat/Commentaires\\
		\hline
		MdstC.1 & 3.1.1.7.2.1 3.1.1.7.2.2 3.1.1.7.2.3 & Ouvrir le module de statistique. & Une partie textuelle apparait dans la partie haute de la fenêtre et une partie graphique dans la partie basse. & \\
		\cline{1-2}\cline{4-5}
		MdstC.1.1 & 3.1.1.7.1.2.1 3.1.1.7.2.4 & & Le nom du fichier de l'utilisateur le plus téléchargé apparait dans la partie textuelle. & \\
		\cline{1-2}\cline{4-5}
		MdstC.1.2 & 3.1.1.7.1.2.2 3.1.1.7.2.5 & & L'identifiant de la personne qui a téléchargé le plus de fichier de l'utilisateur apparait dans la partie textuelle. & \\
		\cline{1-2}\cline{4-5}
		MdstC.1.3 & 3.1.1.7.1.1 3.1.1.7.1.2.3 3.1.1.7.2.6 & & Un histogramme représentant le nombre de téléchargement de chaque fichier apparait dans la partie graphique. & \\
		\cline{1-2}\cline{4-5}
		MdstC.1.3.1 & 3.1.1.7.2.7 3.1.1.7.2.8 3.1.1.7.2.9 & & Une barre s'affiche dans l'histogramme pour chaque fichier de l'utilisateur. & \\
		\cline{1-2}\cline{4-5}
		MdstC.1.3.2 & 3.1.1.7.2.10 & & Chaque barre possède une couleur différente. & \\
		\cline{1-2}\cline{4-5}
		MdstC.1.3.3 & 3.1.1.7.2.11 & & La hauteur des barres correspond au nombre de téléchargement de chaque fichier. & \\
		\cline{1-2}\cline{4-5}
		MdstC.1.3.4 & 3.1.1.7.2.12 & & La largeur des barres est adaptée pour que l'histogramme prenne entièrement la largeur de la page. & \\
		\cline{1-2}\cline{4-5}
		MdstC.1.3.5 & 3.1.1.7.2.13 & & En ordonnée le nombre de fichier est affiché. & \\
		\cline{1-2}\cline{4-5}
		MdstC.1.3.6 & 3.1.1.7.2.14 & & En abscisse un nombre est affiché sous chacune des barres. & \\
		\cline{1-2}\cline{4-5}
		MdstC.1.3.6.1 & 3.1.1.7.2.15 & & Une légende est affichée avec la correspondance numéro de barre/nom du fichier. & \\
		\hline	
	\end{tabular}
	\egroup
\end{center}







\section{Tests Serveur}
\subsection{Authentification}
\begin{center}
	\bgroup
	\def\arraystretch{1.5}
	\begin{tabular}{|p{1.5cm}|p{2cm}|p{8.5cm}|p{8.5cm}|p{5cm}|}
		\hline
		\rowcolor{gris}Référence & Exigence & Test & Résultat attendu & Résultat/Commentaires\\
		\hline
		AuthS.1 & 3.1.2.1.2.1 & Envoyer une chaine de caractère contenant un pseudo suivit d'un mot de passe & Le programme détecte cette arrivé et l'indique dans une console de log & \\
		\hline		
		AuthS.2 & 3.1.2.1.2.2 & Après avoir récupéré les informations essayer de les faires correspondre a la liste des pseudo / Mot de passe. & Le programme indique dans une console de log le succès ou l'échec de l'opération & \\
		\hline		
		AuthS.3 & 3.1.2.1.2.3 & Envoyer une chaine contenant le mot "succès" ou "échec" (suivant la réponse au test AuthS.2) sur un client sommaire. & Chaine reçu en concordance avec le résultat du test AuthS.2 & \\
		\hline		
		AuthS.4 & 3.1.2.1.2.4 & & & \\
		\hline
		AuthS.5 & 3.1.2.1.2.5 & Se connecter en temps qu'Administrateur. &  Avoir accès a la totalité des fichiers du serveur. & \\
		\hline
	\end{tabular}
	\egroup
\end{center}

\subsection{Partage de fichier}
\begin{center}
	\bgroup
	\def\arraystretch{1.5}
	\begin{tabular}{|p{1.5cm}|p{2cm}|p{8.5cm}|p{8.5cm}|p{5cm}|}
		\hline
		\rowcolor{gris}Référence & Exigence & Test & Résultat attendu & Résultat/Commentaires\\
		\hline
		& & & & \\
		\hline
	\end{tabular}
	\egroup
\end{center}

\subsection{Gestionnaire de comptes utilisateurs}
\begin{center}
	\bgroup
	\def\arraystretch{1.5}
	\begin{tabular}{|p{1.5cm}|p{2cm}|p{8.5cm}|p{8.5cm}|p{5cm}|}
		\hline
		\rowcolor{gris}Référence & Exigence & Test & Résultat attendu & Résultat/Commentaires\\
		\hline
		& & & & \\
		\hline
	\end{tabular}
	\egroup
\end{center}

\subsection{Stockage des comptes}
\begin{center}
	\bgroup
	\def\arraystretch{1.5}
	\begin{tabular}{|p{1.5cm}|p{2cm}|p{8.5cm}|p{8.5cm}|p{5cm}|}
		\hline
		\rowcolor{gris}Référence & Exigence & Test & Résultat attendu & Résultat/Commentaires\\
		\hline
		& & & & \\
		\hline
	\end{tabular}
	\egroup
\end{center}



\end{document}