\documentclass[10pt,a4paper,landscape]{report}

\usepackage[utf8]{inputenc}
\usepackage{amsmath}
\usepackage{amsfonts}
\usepackage{amssymb}
\usepackage{graphicx}
\usepackage{color}
\definecolor{gris}{rgb}{0.75,0.75,0.75}
\usepackage{colortbl}
\usepackage{enumitem}
\usepackage{version}
\usepackage{multirow}
\usepackage[top=1cm, bottom=2cm, left=2cm, right=2cm]{geometry}

\usepackage{fancyhdr}
\pagestyle{fancy}

\fancyhead{}
\fancyfoot{} 
\lhead{\includegraphics{../Logo/logoKNKMini.jpg} \hspace{0.1cm} Kould Not Konect  \hspace{0.4cm} \vline}
\chead{Cahier des tests de recette}
\rhead{Kould Not Share}
\rfoot{\thepage}

\author{Nicolas REYNAUD, Kevin BASCOL}
\title{Cahier des tests de recette}
\date{5 Novembre 2014}

\makeatletter
\renewcommand{\thesection}{\@arabic\c@section}
\makeatother

\begin{document}
\makeatletter
	\begin{titlepage}
	
	\begin{figure}
		\begin{minipage}[c]{.46\linewidth}
		\end{minipage} \hfill
		\begin{minipage}[c]{.20\linewidth}
			\begin{center}
				\includegraphics{../Logo/logoKNK.jpg}\\
				{\large Kould Not Konect}
			\end{center}
		\end{minipage}
	\vspace{1cm}
	\end{figure}
	
	\centering
		{
		\hrule height 2pt
		\vspace{0.7cm}
		\Huge \textbf{\@title}}\\
		\vspace{0.7cm}
		\hrule height 2pt
		\vspace{1.5cm}
		{\LARGE  Projet \textbf{Kould Not Share} v1.0}
		
		\vfill
		
		\begin{tabular}{|c|c|c|}
			\hline
			Version & Date & Description\\
			\hline
			v.1 & 05/11/14 & Tests de la première version des exigences\\
			\hline
			 & & \\
			\hline
			 & & \\
			\hline
		\end{tabular}\\
		\vspace{1cm}
		\@author\\
		\end{titlepage}
\makeatother
\setcounter{secnumdepth}{5}
\setcounter{tocdepth}{5}
\renewcommand{\contentsname}{Sommaire}
\begingroup\makeatletter
\def\@makeschapterhead#1{%
  {\parindent \z@ \raggedright
    \normalfont
    \interlinepenalty\@M
    \Huge \bfseries  #1\par\nobreak
    \vskip 20pt% <---- à réduire pour avoir plus de place
  }}\makeatother
\tableofcontents
\endgroup
\thispagestyle{empty}
\setcounter{page}{0}
\newpage

\newgeometry{top=2cm, bottom=2cm, left=0.5cm, right=0.5cm}

\section{Tests Client}
\subsection{Authentification}
\begin{center}
\bgroup
\def\arraystretch{1.5}
	\begin{tabular}{|p{1.5cm}|p{2cm}|p{8.5cm}|p{8.5cm}|p{5cm}|}
		\hline
		\rowcolor{gris}Référence & Exigence & Test & Résultat attendu & Résultat/Commentaires\\
		\hline
		AuthC.1 & 3.1.1.1.1.1 3.1.1.1.1.2 3.1.1.1.1.3 3.1.1.1.1.13 & Ouvrir le logiciel. & Le formulaire de connexion apparait en pop-up au centre de la fenêtre, le reste de l'application est grisé (hormis la barre d'action). & \\
		\hline
		AuthC.2 & 3.1.1.1.1.4 3.1.1.1.1.5 & Cliquer sur le bouton de déconnexion alors que le formulaire de connexion est ouvert.& Le bouton doit être grisé et ne doit rien déclencher. & \\
		\hline
		AuthC.3 & 3.1.1.1.1.6 3.1.1.1.1.7 3.1.1.1.1.8 & Ouvrir le pop-up de connexion. & Le titre "Authentification" apparait centré et est au dessus d'une barre horizontale. & \\
		\cline{1-2}\cline{4-5}
		AuthC.4 & 3.1.1.1.1.9 3.1.1.1.1.10 & & Les mots "Serveur", "Nom d'utilisateur" et "Mot de passe" sont affichés, chacun suivit d'un champ de saisie de texte. & \\
		\cline{1-2}\cline{4-5}
		AuthC.5 & 3.1.1.1.1.11 & & Un bouton "Se connecter" est affiché centré en bas du pop-up & \\
		\hline
		AuthC.6 & 3.1.1.1.1.11 & Dans le pop-up de connexion, remplir incorrectement un ou plusieurs champs de saisie puis cliquer sur le bouton "Se connecter" (à effectuer avec plusieurs combinaison de champs incorrects). & Le(s) champ(s) incorrect(s) possède(nt) une bordure rouge. & \\
		\hline
		AthC.7 & 3.1.1.1.2.1 & Dans le pop-up de connexion, remplir correctement les champs, puis cliquer sur "Se connecter". & Le programme se connecte au serveur avec l'identifiant donné. & \\
		\hline
	\end{tabular}
\egroup
\end{center}

\begin{comment}
		\item Le programme doit permettre à un client de se connecter sur le client FTP.
		\item Le nom d'utilisateur et le mot de passe doivent être définis par l'administrateur du serveur.
		\item Le programme doit proposer un champ de texte éditable pour entrer un pseudo.
		\item Le programme doit proposer un champ de texte éditable pour entrer mot de passe.
		\item Le programme doit permettre d'envoyer les données d'authentification via un protocole sécurisé.
		\item Le programme doit envoyer le mot de passe crypté au serveur.
		\item Le programme doit indiquer le succès de la connexion à l'aide la phrase "Bonjour [pseudo]" présente dans la barre des menus.
		\item Le programme doit indiquer si une erreur est survenue et son type.
		\item En cas d'erreur le programme doit indiquer l'erreur et redemander le mot de passe et/ou le login (en fonction de l'erreur survenue)
		\item En cas d'erreur sur l'adresse du serveur le programme doit indiquer l'erreur et redemander l'adresse du serveur.
		\item Le programme doit vérifier que l'adresse du serveur est une adresse IP V4.
		\item Le programme doit vérifier que le nom d'utilisateur ne dépasse pas 30 caractères.
		\item Le programme doit vérifier que le nom d'utilisateur n'a pas moins de 3 caractères.
		\item Le programme doit vérifier que le mot de passe comporte plus de 6 caractères.
		\item Le programme doit vérifier que le mot de passe comporte moins de 50 caractères.
		\begin{enumerate}
			\item Exemple avec la phrase suivante en cas de mauvais mot de passe "Erreur de mot de passe".
			\item Exemple en cas de pseudo non reconnu "L'utilisateur entré est invalide".
		\end{enumerate}
		\item Le programme doit permettre l'auto-connexion au démarrage du programme, pour ce faire l'utilisateur pourra cocher une case.
	\end{enumerate}	
\end{comment}

\subsection{Gestionnaire de marques-page}
\begin{center}
	\begin{tabular}{|c|c|c|c|}
		\hline
		\rowcolor{gris}Référence & Test & Résultat attendu & Résultat/Commentaires\\
		\hline
		 & & & \\
		\hline
		 & & & \\
		\hline
	\end{tabular}
\end{center}

\section{Tests Serveur}
\end{document}